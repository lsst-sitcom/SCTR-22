\documentclass[SE,lsstdraft,STR,toc]{lsstdoc}
\usepackage{geometry}
\usepackage{longtable,booktabs}
\usepackage{enumitem}
\usepackage{arydshln}

\input meta.tex

\providecommand{\tightlist}{
  \setlength{\itemsep}{0pt}\setlength{\parskip}{0pt}}

\setcounter{tocdepth}{4}

\begin{document}

\def\milestoneName{M1M3 Functional Re-Verification And Integration With Sal V4.0}
\def\milestoneId{LVV-P67}
\def\product{SIT-COM Integration}

\setDocCompact{true}

\title{ LVV-P67 M1M3 Functional Re-Verification And Integration With Sal V4.0 Test Plan and Report}
\setDocRef{\lsstDocType-\lsstDocNum}
\date{\vcsdate}
\setDocUpstreamLocation{\url{https://github.com/lsst/lsst-texmf/examples}}
\author{ Bo Xin }

\input history_and_info.tex


\setDocAbstract{
This is the test plan and report for LVV-P67 (M1M3 Functional Re-Verification And Integration With Sal V4.0),
an LSST milestone pertaining to the System Engineering Subsystem.
}


\maketitle

\section{Introduction}
\label{sect:intro}


\subsection{Objectives}
\label{sect:objectives}

 The objective of this test plan is two-fold:

\begin{itemize}
\tightlist
\item
  to re-verify m1m3 software after SAL version being upgraded to 4.0.
  Make sure all commands, events, and telemetry are recorded in the
  summit EFD.
\item
  rerun most of the M1M3 performance tests which showed anomalies after
  the M1M3 {installation on Pachon in October 2019. Make sure
  performance requirements in \citeds{LTS-88} are met.}
\end{itemize}



\subsection{System Overview}
\label{sect:systemoverview}

 M1M3 surrogate is attached to the mirror cell. M1M3 control software has
been upgraded to work with SAL4.0.


\subsection{Document Overview}
\label{sect:docoverview}

This document was generated from Jira, obtaining the relevant information from the 
\href{https://jira.lsstcorp.org/secure/Tests.jspa#/testPlan/LVV-P67}{LVV-P67}
~Jira Test Plan and related Test Cycles (
  \href{https://jira.lsstcorp.org/secure/Tests.jspa#/testCycle/LVV-C126}{LVV-C126}
).

Section \ref{sect:intro} provides an overview of the test campaign, the system under test (\product{}), the applicable documentation, and explains how this document is organized.
Section \ref{sect:configuration}  describes the configuration used for this test.
Section \ref{sect:personnel} describes the necessary roles and lists the individuals assigned to them.
%Section \ref{sect:plannedtestactivities} provides the list of planned test cycles and test cases, including all relevant information that fully describes the test campaign.

Section \ref{sect:overview} provides a summary of the test results, including an overview in Table \ref{table:summary}, an overall assessment statement and suggestions for possible improvements.
Section \ref{sect:detailedtestresults} provides detailed results for each step in each test case.

The current status of test plan LVV-P67 in Jira is \textbf{ Draft }.

\subsection{References}
\label{sect:references}
\renewcommand{\refname}{}
\bibliography{lsst,refs,books,refs_ads,local}
\section{Test Configuration}
\label{sect:configuration}

\subsection{Data Collection}

  Observing is not required for this test campaign.

\subsection{Verification Environment}
\label{sect:hwconf}
  M1M3 will be verified in a climate controlled environment on the summit
3rd floor

  \subsection{Entry Criteria}
  In order to run these tests, the following criteria must be met first:\\

\begin{itemize}
\tightlist
\item
  All utilities and electrical connections are hooked up and allow the
  M1M3 to be powered on and controlled
\item
  The EFD must be set up to be able to store events and telemetry data
\end{itemize}

  \subsection{Exit Criteria}
  In order for this event to be considered complete, the following
criteria must be met:

\begin{itemize}
\tightlist
\item
  Raw test data, events, and telemetry have been saved for M1M3.
\item
  All test data has been analyzed and post processed.
\item
  All test steps have been statused in the Jira Test Cases within this
  Test Plan and actual results populated as required.
\item
  A summary of the results of the test campaign has been captured in the
  Overall Assessment and Recommended Improvements fields of this Test
  Plan
\item
  A link to the verification artifacts used to produce the summary of
  results has been populated in the Verification Artifacts field of this
  Test Plan
\item
  Any failures have been captured in the
  \href{https://jira.lsstcorp.org/projects/FRACAS/issues/}{FRACAS}
  project
\end{itemize}

  \subsection{PMCS Activity}
  See Epics in Traceability Tab

\newpage
\section{Personnel}
\label{sect:personnel}

The personnel involved in the test campaign is shown in the following table.

\begin{longtable}{p{3cm}p{3cm}p{3cm}p{6cm}}
\hline
\multicolumn{2}{r}{Test Plan (LVV-P67) owner:} &
\multicolumn{2}{l}{\textbf{ Bo Xin } }\\\hline
\multicolumn{2}{r}{ LVV-C126 owner:} &
\multicolumn{2}{l}{\textbf{
    Bo Xin
}
} \\\hline
\textbf{Test Case} & \textbf{Assigned to} & \textbf{Executed by} & \textbf{Additional Test Personnel} \\ \hline
\href{https://jira.lsstcorp.org/secure/Tests.jspa#/testCase/LVV-T247}{LVV-T247}
& {\small Bo Xin } & {\small  } &
\begin{minipage}[]{6cm}
\smallskip
{\small  }
\medskip
\end{minipage}
\\ \hline
\href{https://jira.lsstcorp.org/secure/Tests.jspa#/testCase/LVV-T1761}{LVV-T1761}
& {\small Bo Xin } & {\small Bo Xin } &
\begin{minipage}[]{6cm}
\smallskip
{\small  }
\medskip
\end{minipage}
\\ \hline
\href{https://jira.lsstcorp.org/secure/Tests.jspa#/testCase/LVV-T253}{LVV-T253}
& {\small Bo Xin } & {\small Bo Xin } &
\begin{minipage}[]{6cm}
\smallskip
{\small  }
\medskip
\end{minipage}
\\ \hline
\href{https://jira.lsstcorp.org/secure/Tests.jspa#/testCase/LVV-T1801}{LVV-T1801}
& {\small Bo Xin } & {\small Bo Xin } &
\begin{minipage}[]{6cm}
\smallskip
{\small  }
\medskip
\end{minipage}
\\ \hline
\href{https://jira.lsstcorp.org/secure/Tests.jspa#/testCase/LVV-T1809}{LVV-T1809}
& {\small Bo Xin } & {\small Bo Xin } &
\begin{minipage}[]{6cm}
\smallskip
{\small  }
\medskip
\end{minipage}
\\ \hline
\href{https://jira.lsstcorp.org/secure/Tests.jspa#/testCase/LVV-T228}{LVV-T228}
& {\small Bo Xin } & {\small  } &
\begin{minipage}[]{6cm}
\smallskip
{\small  }
\medskip
\end{minipage}
\\ \hline
\href{https://jira.lsstcorp.org/secure/Tests.jspa#/testCase/LVV-T239}{LVV-T239}
& {\small Bo Xin } & {\small  } &
\begin{minipage}[]{6cm}
\smallskip
{\small  }
\medskip
\end{minipage}
\\ \hline
\href{https://jira.lsstcorp.org/secure/Tests.jspa#/testCase/LVV-T235}{LVV-T235}
& {\small Bo Xin } & {\small  } &
\begin{minipage}[]{6cm}
\smallskip
{\small  }
\medskip
\end{minipage}
\\ \hline
\href{https://jira.lsstcorp.org/secure/Tests.jspa#/testCase/LVV-T234}{LVV-T234}
& {\small Bo Xin } & {\small  } &
\begin{minipage}[]{6cm}
\smallskip
{\small  }
\medskip
\end{minipage}
\\ \hline
\href{https://jira.lsstcorp.org/secure/Tests.jspa#/testCase/LVV-T254}{LVV-T254}
& {\small ahautmann } & {\small  } &
\begin{minipage}[]{6cm}
\smallskip
{\small  }
\medskip
\end{minipage}
\\ \hline
\href{https://jira.lsstcorp.org/secure/Tests.jspa#/testCase/LVV-T229}{LVV-T229}
& {\small Bo Xin } & {\small  } &
\begin{minipage}[]{6cm}
\smallskip
{\small  }
\medskip
\end{minipage}
\\ \hline
\href{https://jira.lsstcorp.org/secure/Tests.jspa#/testCase/LVV-T238}{LVV-T238}
& {\small Bo Xin } & {\small  } &
\begin{minipage}[]{6cm}
\smallskip
{\small  }
\medskip
\end{minipage}
\\ \hline
\href{https://jira.lsstcorp.org/secure/Tests.jspa#/testCase/LVV-T230}{LVV-T230}
& {\small Bo Xin } & {\small  } &
\begin{minipage}[]{6cm}
\smallskip
{\small  }
\medskip
\end{minipage}
\\ \hline
\href{https://jira.lsstcorp.org/secure/Tests.jspa#/testCase/LVV-T231}{LVV-T231}
& {\small Bo Xin } & {\small  } &
\begin{minipage}[]{6cm}
\smallskip
{\small  }
\medskip
\end{minipage}
\\ \hline
\href{https://jira.lsstcorp.org/secure/Tests.jspa#/testCase/LVV-T232}{LVV-T232}
& {\small Bo Xin } & {\small  } &
\begin{minipage}[]{6cm}
\smallskip
{\small  }
\medskip
\end{minipage}
\\ \hline
\href{https://jira.lsstcorp.org/secure/Tests.jspa#/testCase/LVV-T241}{LVV-T241}
& {\small Bo Xin } & {\small  } &
\begin{minipage}[]{6cm}
\smallskip
{\small  }
\medskip
\end{minipage}
\\ \hline
\href{https://jira.lsstcorp.org/secure/Tests.jspa#/testCase/LVV-T242}{LVV-T242}
& {\small Bo Xin } & {\small  } &
\begin{minipage}[]{6cm}
\smallskip
{\small  }
\medskip
\end{minipage}
\\ \hline
\href{https://jira.lsstcorp.org/secure/Tests.jspa#/testCase/LVV-T237}{LVV-T237}
& {\small Bo Xin } & {\small  } &
\begin{minipage}[]{6cm}
\smallskip
{\small  }
\medskip
\end{minipage}
\\ \hline
\href{https://jira.lsstcorp.org/secure/Tests.jspa#/testCase/LVV-T236}{LVV-T236}
& {\small Bo Xin } & {\small  } &
\begin{minipage}[]{6cm}
\smallskip
{\small  }
\medskip
\end{minipage}
\\ \hline
\href{https://jira.lsstcorp.org/secure/Tests.jspa#/testCase/LVV-T257}{LVV-T257}
& {\small Bo Xin } & {\small  } &
\begin{minipage}[]{6cm}
\smallskip
{\small  }
\medskip
\end{minipage}
\\ \hline
\href{https://jira.lsstcorp.org/secure/Tests.jspa#/testCase/LVV-T245}{LVV-T245}
& {\small Bo Xin } & {\small  } &
\begin{minipage}[]{6cm}
\smallskip
{\small  }
\medskip
\end{minipage}
\\ \hline
\end{longtable}

\newpage

\section{Test Campaign Overview}
\label{sect:overview}

\subsection{Summary}
\label{sect:summarytable}

\begin{longtable}{p{2cm}p{2.5cm}p{9cm}p{2.5cm}}
\toprule
\multicolumn{3}{p{13.5cm}}{ Test Plan {\bf LVV-P67: M1M3 Functional Re-verification and Integration with SAL v4.0 }} & Draft \\\hline

  \multicolumn{3}{p{13.5cm}}{ Test Cycle {\bf LVV-C126: M1M3 Functional Re-verification and Integration with SAL v4.0 }} & In Progress \\\hline

  {\bf \footnotesize test case} & {\bf \footnotesize status} & {\bf \footnotesize comment} & {\bf \footnotesize issues} \\\toprule

    \href{https://jira.lsstcorp.org/secure/Tests.jspa#/testCase/LVV-T247}{LVV-T247}
    & Not Executed &
    \begin{minipage}[]{9cm}
    \smallskip
    
    \medskip
    \end{minipage}
    &
    \\\hline
    \href{https://jira.lsstcorp.org/secure/Tests.jspa#/testCase/LVV-T1761}{LVV-T1761}
    & Blocked &
    \begin{minipage}[]{9cm}
    \smallskip
    The mirror door is not in place yet. It is on its way to the summit. So
this cannot be executed at this time.
    \medskip
    \end{minipage}
    &
    \\\hline
    \href{https://jira.lsstcorp.org/secure/Tests.jspa#/testCase/LVV-T253}{LVV-T253}
    & Blocked &
    \begin{minipage}[]{9cm}
    \smallskip
    Thermal system not yet installed. Felipe says it is going to be later
this year, most likely in May.
    \medskip
    \end{minipage}
    &
    \\\hline
    \href{https://jira.lsstcorp.org/secure/Tests.jspa#/testCase/LVV-T1801}{LVV-T1801}
    & Pass &
    \begin{minipage}[]{9cm}
    \smallskip
    Note that the initial forces are different from the LUT table in the
github repo, because the actual LUT being used are those named like
*\_2019xxxxa.csv. They cannot be found in the GitHub repo.
    \medskip
    \end{minipage}
    &
    \\\hline
    \href{https://jira.lsstcorp.org/secure/Tests.jspa#/testCase/LVV-T1809}{LVV-T1809}
    & Pass &
    \begin{minipage}[]{9cm}
    \smallskip
    can we get mx my fx fy all close to zero at the same time? No
necessarily. fx=fy=0 is what we need, so that the axial actuators are
all upright. If the mass distribution is uneven, we won't get mx=my=0.
It is OK.
    \medskip
    \end{minipage}
    &
    \\\hline
    \href{https://jira.lsstcorp.org/secure/Tests.jspa#/testCase/LVV-T228}{LVV-T228}
    & Not Executed &
    \begin{minipage}[]{9cm}
    \smallskip
    
    \medskip
    \end{minipage}
    &
    \\\hline
    \href{https://jira.lsstcorp.org/secure/Tests.jspa#/testCase/LVV-T239}{LVV-T239}
    & Not Executed &
    \begin{minipage}[]{9cm}
    \smallskip
    
    \medskip
    \end{minipage}
    &
    \\\hline
    \href{https://jira.lsstcorp.org/secure/Tests.jspa#/testCase/LVV-T235}{LVV-T235}
    & Not Executed &
    \begin{minipage}[]{9cm}
    \smallskip
    
    \medskip
    \end{minipage}
    &
    \\\hline
    \href{https://jira.lsstcorp.org/secure/Tests.jspa#/testCase/LVV-T234}{LVV-T234}
    & Not Executed &
    \begin{minipage}[]{9cm}
    \smallskip
    
    \medskip
    \end{minipage}
    &
    \\\hline
    \href{https://jira.lsstcorp.org/secure/Tests.jspa#/testCase/LVV-T254}{LVV-T254}
    & Not Executed &
    \begin{minipage}[]{9cm}
    \smallskip
    
    \medskip
    \end{minipage}
    &
    \\\hline
    \href{https://jira.lsstcorp.org/secure/Tests.jspa#/testCase/LVV-T229}{LVV-T229}
    & Not Executed &
    \begin{minipage}[]{9cm}
    \smallskip
    
    \medskip
    \end{minipage}
    &
    \\\hline
    \href{https://jira.lsstcorp.org/secure/Tests.jspa#/testCase/LVV-T238}{LVV-T238}
    & Not Executed &
    \begin{minipage}[]{9cm}
    \smallskip
    
    \medskip
    \end{minipage}
    &
    \\\hline
    \href{https://jira.lsstcorp.org/secure/Tests.jspa#/testCase/LVV-T230}{LVV-T230}
    & Not Executed &
    \begin{minipage}[]{9cm}
    \smallskip
    
    \medskip
    \end{minipage}
    &
    \\\hline
    \href{https://jira.lsstcorp.org/secure/Tests.jspa#/testCase/LVV-T231}{LVV-T231}
    & Not Executed &
    \begin{minipage}[]{9cm}
    \smallskip
    
    \medskip
    \end{minipage}
    &
    \\\hline
    \href{https://jira.lsstcorp.org/secure/Tests.jspa#/testCase/LVV-T232}{LVV-T232}
    & Not Executed &
    \begin{minipage}[]{9cm}
    \smallskip
    
    \medskip
    \end{minipage}
    &
    \\\hline
    \href{https://jira.lsstcorp.org/secure/Tests.jspa#/testCase/LVV-T241}{LVV-T241}
    & Not Executed &
    \begin{minipage}[]{9cm}
    \smallskip
    
    \medskip
    \end{minipage}
    &
    \\\hline
    \href{https://jira.lsstcorp.org/secure/Tests.jspa#/testCase/LVV-T242}{LVV-T242}
    & Not Executed &
    \begin{minipage}[]{9cm}
    \smallskip
    
    \medskip
    \end{minipage}
    &
    \\\hline
    \href{https://jira.lsstcorp.org/secure/Tests.jspa#/testCase/LVV-T237}{LVV-T237}
    & Not Executed &
    \begin{minipage}[]{9cm}
    \smallskip
    
    \medskip
    \end{minipage}
    &
    \\\hline
    \href{https://jira.lsstcorp.org/secure/Tests.jspa#/testCase/LVV-T236}{LVV-T236}
    & Not Executed &
    \begin{minipage}[]{9cm}
    \smallskip
    
    \medskip
    \end{minipage}
    &
    \\\hline
    \href{https://jira.lsstcorp.org/secure/Tests.jspa#/testCase/LVV-T257}{LVV-T257}
    & Not Executed &
    \begin{minipage}[]{9cm}
    \smallskip
    
    \medskip
    \end{minipage}
    &
    \\\hline
    \href{https://jira.lsstcorp.org/secure/Tests.jspa#/testCase/LVV-T245}{LVV-T245}
    & Not Executed &
    \begin{minipage}[]{9cm}
    \smallskip
    
    \medskip
    \end{minipage}
    &
    \\\hline
\caption{Test Campaign Summary}
\label{table:summary}
\end{longtable}

\subsection{Overall Assessment}
\label{sect:overallassessment}

Not yet available.

\subsection{Recommended Improvements}
\label{sect:recommendations}

Not yet available.

\newpage
\section{Detailed Test Results}
\label{sect:detailedtestresults}

\subsection{Test Cycle LVV-C126 }

Open test cycle {\it \href{https://jira.lsstcorp.org/secure/Tests.jspa#/testrun/LVV-C126}{M1M3 Functional Re-verification and Integration with SAL v4.0}} in Jira.

M1M3 Functional Re-verification and Integration with SAL v4.0\\
Status: In Progress



\subsubsection{Software Version/Baseline}
Not provided.

\subsubsection{Configuration}
Not provided.

\subsubsection{Test Cases in LVV-C126 Test Cycle}

\paragraph{Test Case LVV-T247 - M13T-036 Environmental Conditions Operations }\mbox{}\\

Open  \href{https://jira.lsstcorp.org/secure/Tests.jspa#/testCase/LVV-T247}{\textit{ LVV-T247 } }
test case in Jira.

The overall environmental constraints are listed in \citeds{LTS-54}. This set of
requirements from \citeds{LTS-88} is a flow-down from \citeds{LTS-54}. These tests will
verify that the M1M3 system can operate under the conditions found at
CAID and SOML ahead of being shipped to Pachon.

\textbf{ Preconditions}:\\
\begin{enumerate}
\tightlist
\item
  All M1M3 Systems installed, except thermal control and interlocks
\item
  All M1M3 Systems operating, except thermal control and interlocks
\item
  The accelerometers on the mirror are operational and the outputs are
  being recorded.
\item
  The independent measuring system (IMS) is operational and the outputs
  are being recorded.
\item
  The M1M3 mirror system is in active mode and the mirror is resting on
  the actuators
\item
  The M1M3 mirror system is in passive mode and the mirror is resting on
  the static supports.
\item
  The M1M3 mirror positioning system is operational and the outputs are
  being recorded.
\item
  Hard points are in their nominal positions with negligible loads
  present.
\item
  M1M3 control software is current and configured for this test.
\item
  All test data is being recorded to the Engineering Facility Database
  EFD
\end{enumerate}

Execution status: {\bf Not Executed }

Final comment:\\


Detailed steps results:

\begin{longtable}{p{1cm}p{15cm}}
\hline
{Step} & Step Details\\ \hline
1 & Description \\
 & \begin{minipage}[t]{15cm}
{\footnotesize
\begin{enumerate}
\tightlist
\item
  During all of the testing on the summit, compile the environmental
  conditions (if they are available):

  \begin{enumerate}
  \tightlist
  \item
    Humidity
  \item
    Temperature
  \item
    Wind Speed
  \item
    Pressure~
  \item
    etc.~
  \end{enumerate}
\end{enumerate}

\medskip }
\end{minipage}
\\ \cdashline{2-2}

 & Test Data \\
 & \begin{minipage}[t]{15cm}{\footnotesize
Log file of environmental data\\[2\baselineskip]

\medskip }
\end{minipage} \\ \cdashline{2-2}

 & Expected Result \\
 & \begin{minipage}[t]{15cm}{\footnotesize
\begin{itemize}
\tightlist
\item
  Demonstrate that the mirror support system is fully functional on the
  summit.
\item
  Mirror support system is functional over the range of temperatures
  ~during the testing up to 20C and not lower than -5 C
\item
  Mirror support system is functional over range of wind speeds not
  greater than 5 m/s.
\end{itemize}

\medskip }
\end{minipage} \\ \cdashline{2-2}

 & Actual Result \\
 & \begin{minipage}[t]{15cm}{\footnotesize

\medskip }
\end{minipage} \\ \cdashline{2-2}

 & Status: \textbf{ Not Executed } \\ \hline

\end{longtable}

\paragraph{Test Case LVV-T1761 - M13T-INT-002: Mirror Cell Door Open Interlock }\mbox{}\\

Open  \href{https://jira.lsstcorp.org/secure/Tests.jspa#/testCase/LVV-T1761}{\textit{ LVV-T1761 } }
test case in Jira.

\begin{longtable}[]{@{}ll@{}}
\toprule
The purpose of this test is verify the correct operation of the
electrical hardware associated to the interlock system &\tabularnewline
\bottomrule
\end{longtable}

\textbf{ Preconditions}:\\
-Factory test done.\\
-Safety controller powered.\\
-Safety controller M1M3 IS firmware running.\\
-M1M3 IS User Interface available\\
-Support Controller On\\
-Thermal Controller On\\
-GIS simulator present (if not bypass ``GIS Heartbeat loss'' event)\\
-No events or actions triggered

Execution status: {\bf Blocked }

Final comment:\\The mirror door is not in place yet. It is on its way to the summit. So
this cannot be executed at this time.


Detailed steps results:

\begin{longtable}{p{1cm}p{15cm}}
\hline
{Step} & Step Details\\ \hline
1 & Description \\
 & \begin{minipage}[t]{15cm}
{\footnotesize
open mirror cell door

\medskip }
\end{minipage}
\\ \cdashline{2-2}


 & Expected Result \\
 & \begin{minipage}[t]{15cm}{\footnotesize
\begin{itemize}
\tightlist
\item
  ``Mirror door open'' indicator active at user interface
\item
  ``TMA Motion Block'' action indicator active at user interface
\item
  M1M3 IS output

  \begin{itemize}
  \tightlist
  \item
    main.4.O2(14) off??
  \end{itemize}
\item
  Support controller detects ``Block TMA Motion'' action status
\end{itemize}

If test fails check: ??

\begin{itemize}
\tightlist
\item
  lights power is present, power should be 24V. If not check power
  supply P11, relays K13, K18
\item
  M1M3 IS input main.1.I0(11) voltage, if event is present should be 0V,
  if not present 24V.
\end{itemize}

\medskip }
\end{minipage} \\ \cdashline{2-2}

 & Actual Result \\
 & \begin{minipage}[t]{15cm}{\footnotesize

\medskip }
\end{minipage} \\ \cdashline{2-2}

 & Status: \textbf{ Not Executed } \\ \hline

2 & Description \\
 & \begin{minipage}[t]{15cm}
{\footnotesize
Recovery Process:\\[2\baselineskip]

\begin{itemize}
\tightlist
\item
  close mirror cell door
\item
  reset ``mirror cell door open'' event
\end{itemize}

\medskip }
\end{minipage}
\\ \cdashline{2-2}


 & Expected Result \\
 & \begin{minipage}[t]{15cm}{\footnotesize
\begin{itemize}
\tightlist
\item
  reset ``mirror cell door open'' event~
\end{itemize}

\medskip }
\end{minipage} \\ \cdashline{2-2}

 & Actual Result \\
 & \begin{minipage}[t]{15cm}{\footnotesize

\medskip }
\end{minipage} \\ \cdashline{2-2}

 & Status: \textbf{ Not Executed } \\ \hline

\end{longtable}

\paragraph{Test Case LVV-T253 - M13T-INT-006: Thermal System Cabinet Door Open }\mbox{}\\

Open  \href{https://jira.lsstcorp.org/secure/Tests.jspa#/testCase/LVV-T253}{\textit{ LVV-T253 } }
test case in Jira.

The purpose of this test is verify the correct operation of the
electrical hardware associated to the interlock system

\textbf{ Preconditions}:\\
-Factory test done.\\
-Safety controller powered.\\
-Safety controller M1M3 IS firmware running.\\
-M1M3 IS User Interface available\\
-Support Controller On\\
-Thermal Controller On\\
-GIS simulator present (if not bypass ``GIS Heartbeat loss'' event)\\
-No events or actions triggered\\
-M1M3 mirror unparked

Execution status: {\bf Blocked }

Final comment:\\Thermal system not yet installed. Felipe says it is going to be later
this year, most likely in May.


Detailed steps results:

\begin{longtable}{p{1cm}p{15cm}}
\hline
{Step} & Step Details\\ \hline
1 & Description \\
 & \begin{minipage}[t]{15cm}
{\footnotesize
Close thermal system cabinet doors\\
Confirm Interlock system shows doors are closed and there is no bypass
on the Interlock System

\medskip }
\end{minipage}
\\ \cdashline{2-2}


 & Expected Result \\
 & \begin{minipage}[t]{15cm}{\footnotesize

\medskip }
\end{minipage} \\ \cdashline{2-2}

 & Actual Result \\
 & \begin{minipage}[t]{15cm}{\footnotesize

\medskip }
\end{minipage} \\ \cdashline{2-2}

 & Status: \textbf{ Not Executed } \\ \hline

2 & Description \\
 & \begin{minipage}[t]{15cm}
{\footnotesize
Open left thermal system cabinet door

\medskip }
\end{minipage}
\\ \cdashline{2-2}


 & Expected Result \\
 & \begin{minipage}[t]{15cm}{\footnotesize
-``thermal system cabinet door open'' indicator active at user
interface\\
-``TMA Motion Block'' action indicator active at user interface\\
-M1M3 IS output\\

\begin{itemize}
\tightlist
\item
  ext.3.O2(14) off
\item
  main.4.O2(14) off
\end{itemize}

-Support controller detects ``TMA Motion Block'' action
status\\[2\baselineskip]if test fails check:\\
-M1M3 thermal system door switch\\
-M1M3 IS inputs:\\

\begin{itemize}
\tightlist
\item
  ext.1.I2(14)
\item
  ext.1.I3(24)
\end{itemize}

\medskip }
\end{minipage} \\ \cdashline{2-2}

 & Actual Result \\
 & \begin{minipage}[t]{15cm}{\footnotesize

\medskip }
\end{minipage} \\ \cdashline{2-2}

 & Status: \textbf{ Not Executed } \\ \hline

3 & Description \\
 & \begin{minipage}[t]{15cm}
{\footnotesize
Close left thermal system cabinet door. Reset interlock. Repeat for
right door.~

\medskip }
\end{minipage}
\\ \cdashline{2-2}


 & Expected Result \\
 & \begin{minipage}[t]{15cm}{\footnotesize
-``thermal system cabinet door open'' indicator active at user
interface\\
-``TMA Motion Block'' action indicator active at user interface\\
-M1M3 IS output\\

\begin{itemize}
\tightlist
\item
  ext.3.O2(14) off
\item
  main.4.O2(14) off
\end{itemize}

-Support controller detects ``TMA Motion Block'' action
status\\[2\baselineskip]if test fails check:\\
-M1M3 thermal system door switch\\
-M1M3 IS inputs:\\

\begin{itemize}
\tightlist
\item
  ext.1.I2(14)
\item
  ext.1.I3(24)
\end{itemize}

\medskip }
\end{minipage} \\ \cdashline{2-2}

 & Actual Result \\
 & \begin{minipage}[t]{15cm}{\footnotesize

\medskip }
\end{minipage} \\ \cdashline{2-2}

 & Status: \textbf{ Not Executed } \\ \hline

4 & Description \\
 & \begin{minipage}[t]{15cm}
{\footnotesize
To recover:\\
-close the support system cabinet door\\
-reset ``thermal system cabinet door open'' event

\medskip }
\end{minipage}
\\ \cdashline{2-2}


 & Expected Result \\
 & \begin{minipage}[t]{15cm}{\footnotesize

\medskip }
\end{minipage} \\ \cdashline{2-2}

 & Actual Result \\
 & \begin{minipage}[t]{15cm}{\footnotesize

\medskip }
\end{minipage} \\ \cdashline{2-2}

 & Status: \textbf{ Not Executed } \\ \hline

\end{longtable}

\paragraph{Test Case LVV-T1801 - verify M1M3 SAL 4.0 telemetry against GUI display }\mbox{}\\

Open  \href{https://jira.lsstcorp.org/secure/Tests.jspa#/testCase/LVV-T1801}{\textit{ LVV-T1801 } }
test case in Jira.



\textbf{ Preconditions}:\\


Execution status: {\bf Pass }

Final comment:\\Note that the initial forces are different from the LUT table in the
github repo, because the actual LUT being used are those named like
*\_2019xxxxa.csv. They cannot be found in the GitHub repo.


Detailed steps results:

\begin{longtable}{p{1cm}p{15cm}}
\hline
{Step} & Step Details\\ \hline
1 & Description \\
 & \begin{minipage}[t]{15cm}
{\footnotesize
start control system, enter enabled state, raise the mirror

\medskip }
\end{minipage}
\\ \cdashline{2-2}


 & Expected Result \\
 & \begin{minipage}[t]{15cm}{\footnotesize

\medskip }
\end{minipage} \\ \cdashline{2-2}

 & Actual Result \\
 & \begin{minipage}[t]{15cm}{\footnotesize
started 2/28 2:50 local.\\
connecting to SAL, please wait.\\
3:07pm it is done. Took about 20 minutes.\\
start, enable, raise mirror~

\medskip }
\end{minipage} \\ \cdashline{2-2}

 & Status: \textbf{ Pass } \\ \hline

2 & Description \\
 & \begin{minipage}[t]{15cm}
{\footnotesize
observe all the information in the GUI, make sure they make sense. Take
screen shots\\
Make sure telemetry and events are showing up in EFD.\\
Use Chronograph to check telemetry and events, make sure they agree with
GUI.

\medskip }
\end{minipage}
\\ \cdashline{2-2}


 & Expected Result \\
 & \begin{minipage}[t]{15cm}{\footnotesize

\medskip }
\end{minipage} \\ \cdashline{2-2}

 & Actual Result \\
 & \begin{minipage}[t]{15cm}{\footnotesize
checked that the start, enable, and raise commands were registered in
EFD at 18:09 utc\\
The fx and fy forces are \textasciitilde{}+/- 5 N, as calculated from
actuator weights. Good. As shown on GUI. Same in Chronograf.\\
checked z force for act101, \textasciitilde{}581N, same on GUI and EFD.
z-force for act102 \textasciitilde{}964\\
check the force set against optimized forces, time is3:35pm local.
Notebook is here
https://github.com/bxin/M1M3\_summit\_2003/blob/master/initial\_forces.ipynb\\
compared HP forces. They match.\\
Question:when I first raise the mirror I see net z force on the hard
points of 2500N. Is the FB (force balance) enabled? if not, how do I do
that?\\
Answer: enter engineering state, then you have the button to enable hard
point correction.\\
It is not automatic in raised state. There is a SAL command to enable
that too.\\
checked HP position of course it is 0 everywhere.\\
It is 4:00pm. shutting down the system. will be back on Monday.\\
exit engineering, lower mirror, disable (hear the air), shutdown (hear
valve close). cut power to 4 subnets and lights (switch \#1-4,6), page
Christian to shut compressor.

\medskip }
\end{minipage} \\ \cdashline{2-2}

 & Status: \textbf{ Pass } \\ \hline

3 & Description \\
 & \begin{minipage}[t]{15cm}
{\footnotesize
add 20N to actuator 103, repeat the above

\medskip }
\end{minipage}
\\ \cdashline{2-2}


 & Expected Result \\
 & \begin{minipage}[t]{15cm}{\footnotesize

\medskip }
\end{minipage} \\ \cdashline{2-2}

 & Actual Result \\
 & \begin{minipage}[t]{15cm}{\footnotesize
10:37am local 3/2 Monday. Just raised the mirror.\\
10:45am checked hard points having net z of 2500N. now enable hard point
correction. net z force goes to close to zero.\\
act 103 \textasciitilde{}1031N. 10:48am, added 20N, GUI show
1050N.\\[3\baselineskip]

\medskip }
\end{minipage} \\ \cdashline{2-2}

 & Status: \textbf{ Pass } \\ \hline

\end{longtable}

\paragraph{Test Case LVV-T1809 - calibrate M1M3 optimal position and IMS offset }\mbox{}\\

Open  \href{https://jira.lsstcorp.org/secure/Tests.jspa#/testCase/LVV-T1809}{\textit{ LVV-T1809 } }
test case in Jira.

The hardpoints are numbered and go in specific locations. However, we
had trouble with the hard point extensions and a couple of them had the
extensions replaced. The extensions probably vary in length by
\textasciitilde{} 1mm.\\
The IMS is essentially a relative motion sensor. Setting the readings to
zero when the mirror is at reference position will help avoid some
errors/anomalies in the testing scripts for other test cases.

\textbf{ Preconditions}:\\


Execution status: {\bf Pass }

Final comment:\\can we get mx my fx fy all close to zero at the same time? No
necessarily. fx=fy=0 is what we need, so that the axial actuators are
all upright. If the mass distribution is uneven, we won't get mx=my=0.
It is OK.


Detailed steps results:

\begin{longtable}{p{1cm}p{15cm}}
\hline
{Step} & Step Details\\ \hline
1 & Description \\
 & \begin{minipage}[t]{15cm}
{\footnotesize
raise the mirror, confirm that the optimized z-forces for zenith are
being applied, and the x and y forces are non-zero, as calculated based
on actuator weights.

\medskip }
\end{minipage}
\\ \cdashline{2-2}


 & Expected Result \\
 & \begin{minipage}[t]{15cm}{\footnotesize

\medskip }
\end{minipage} \\ \cdashline{2-2}

 & Actual Result \\
 & \begin{minipage}[t]{15cm}{\footnotesize
done.
see~\href{https://jira.lsstcorp.org/secure/Tests.jspa\#/testPlayer/testExecution/LVV-E1020}{LVV-E1020}

\medskip }
\end{minipage} \\ \cdashline{2-2}

 & Status: \textbf{ Pass } \\ \hline

2 & Description \\
 & \begin{minipage}[t]{15cm}
{\footnotesize
check the balance forces.\\
z force and Mz should be roughly zero, if the Mirror Lab results hold.\\
move the mirror along x and y to make x/y net forces and Mx/My close to
zero.

\medskip }
\end{minipage}
\\ \cdashline{2-2}


 & Expected Result \\
 & \begin{minipage}[t]{15cm}{\footnotesize

\medskip }
\end{minipage} \\ \cdashline{2-2}

 & Actual Result \\
 & \begin{minipage}[t]{15cm}{\footnotesize
11:00am on GUI TAB ``FA - F,M\&E'', we have Fx\textasciitilde{}-158N,
Fy\textasciitilde{}=-21N.\\
Now reposition the mirror5. x-\textgreater{}0.5mm, Fx
\textasciitilde{}-266N.\\
x = -0.5mm, Fx \textasciitilde{}-33N\\
x = -0.6mm, Fx \textasciitilde{}=1N\\
The moments are \textasciitilde{}400 - 500 N*m\\
14:45utc move mirror to x=-0.004m, moments decrease, but fx goes up.\\
we are going to get fx and fy to zero. fx=fy=0 is what we need, so that
the axial actuators are all upright. If the mass distribution is uneven,
we won't get mx=my=0. It is OK.\\
12:04pm local, x=-0.4mm, y=-0.1mm. fx and fy are close to zero now.\\
29694, 29310, 42366, 37704, 27418, 35478

\medskip }
\end{minipage} \\ \cdashline{2-2}

 & Status: \textbf{ Pass } \\ \hline

3 & Description \\
 & \begin{minipage}[t]{15cm}
{\footnotesize
write down the hard point displacements into the configuration file.\\
PositionControllerSettings.xml

\medskip }
\end{minipage}
\\ \cdashline{2-2}


 & Expected Result \\
 & \begin{minipage}[t]{15cm}{\footnotesize

\medskip }
\end{minipage} \\ \cdashline{2-2}

 & Actual Result \\
 & \begin{minipage}[t]{15cm}{\footnotesize
12:16pm edited PositionControllerSettings.xml on control computer. copy
to cRIO.

\medskip }
\end{minipage} \\ \cdashline{2-2}

 & Status: \textbf{ Pass } \\ \hline

4 & Description \\
 & \begin{minipage}[t]{15cm}
{\footnotesize
restart the system, raise the mirror,\\
make sure FB forces and moments are all close to zero now.

\medskip }
\end{minipage}
\\ \cdashline{2-2}


 & Expected Result \\
 & \begin{minipage}[t]{15cm}{\footnotesize

\medskip }
\end{minipage} \\ \cdashline{2-2}

 & Actual Result \\
 & \begin{minipage}[t]{15cm}{\footnotesize
lower the mirror, all the way to ``standby''. Start button will reload
the settings.\\
12:44pm local, confirmed that new setting has been applied.~

\medskip }
\end{minipage} \\ \cdashline{2-2}

 & Status: \textbf{ Pass } \\ \hline

5 & Description \\
 & \begin{minipage}[t]{15cm}
{\footnotesize
write down IMS offsets into the configuration file.\\
DisplacementSensorSettings.xml.

\medskip }
\end{minipage}
\\ \cdashline{2-2}


 & Expected Result \\
 & \begin{minipage}[t]{15cm}{\footnotesize

\medskip }
\end{minipage} \\ \cdashline{2-2}

 & Actual Result \\
 & \begin{minipage}[t]{15cm}{\footnotesize
-0.000258707\\
-0.000050794\\
\hspace*{0.333em}0.000029275\\
0.000004158\\
-0.000093620\\
0.000068720\\
These numbers change a little while we are sitting here and reading
them. So it is OK the last a few digits are now different from the GUI\\
copy to cRIO. restart to load the settings.

\medskip }
\end{minipage} \\ \cdashline{2-2}

 & Status: \textbf{ Pass } \\ \hline

6 & Description \\
 & \begin{minipage}[t]{15cm}
{\footnotesize
restart the system, raise the mirror,\\
make sure the IMS readings are all close to zero now.

\medskip }
\end{minipage}
\\ \cdashline{2-2}


 & Expected Result \\
 & \begin{minipage}[t]{15cm}{\footnotesize

\medskip }
\end{minipage} \\ \cdashline{2-2}

 & Actual Result \\
 & \begin{minipage}[t]{15cm}{\footnotesize
12:55pm lowered the mirror.\\
Note that the above numbers need to be added with what were in the
setting file already to get the new values.\\
restart the system\\
confirmed that the new IMS readings are close to zero. local time
2:00pm. all 6 positions values \textless{}1e-2.\\[2\baselineskip]

\medskip }
\end{minipage} \\ \cdashline{2-2}

 & Status: \textbf{ Pass } \\ \hline

\end{longtable}

\paragraph{Test Case LVV-T228 - M13T-001: M1M3 \& EFD Interface }\mbox{}\\

Open  \href{https://jira.lsstcorp.org/secure/Tests.jspa#/testCase/LVV-T228}{\textit{ LVV-T228 } }
test case in Jira.

This test is intended to verify the interface between the EFD and the
M1M3 Mirror Control system, and that all of the specified data that
needs to be logged for the following tests is verified.~

\textbf{ Preconditions}:\\
\begin{itemize}
\tightlist
\item
  M1M3 Control Software running
\item
  M1M3 Control Software broadcasting to SAL
\item
  EFD running
\item
  EFD is reading from SAL~
\end{itemize}

Execution status: {\bf Not Executed }

Final comment:\\


Detailed steps results:

\begin{longtable}{p{1cm}p{15cm}}
\hline
{Step} & Step Details\\ \hline
1 & Description \\
 & \begin{minipage}[t]{15cm}
{\footnotesize
Initialize EFD and M1M3

\medskip }
\end{minipage}
\\ \cdashline{2-2}


 & Expected Result \\
 & \begin{minipage}[t]{15cm}{\footnotesize
EFD writers all ready to go.\\
M1M3 EUI shows M1M3 ready.~

\medskip }
\end{minipage} \\ \cdashline{2-2}

 & Actual Result \\
 & \begin{minipage}[t]{15cm}{\footnotesize

\medskip }
\end{minipage} \\ \cdashline{2-2}

 & Status: \textbf{ Not Executed } \\ \hline

2 & Description \\
 & \begin{minipage}[t]{15cm}
{\footnotesize
Run the python script

\medskip }
\end{minipage}
\\ \cdashline{2-2}


 & Expected Result \\
 & \begin{minipage}[t]{15cm}{\footnotesize
Generate a separate list of force values and timestamp.~

\medskip }
\end{minipage} \\ \cdashline{2-2}

 & Actual Result \\
 & \begin{minipage}[t]{15cm}{\footnotesize

\medskip }
\end{minipage} \\ \cdashline{2-2}

 & Status: \textbf{ Not Executed } \\ \hline

3 & Description \\
 & \begin{minipage}[t]{15cm}
{\footnotesize
Confirm that the EFD has the same values as output by the python
script.~

\medskip }
\end{minipage}
\\ \cdashline{2-2}


 & Expected Result \\
 & \begin{minipage}[t]{15cm}{\footnotesize
\emph{Zero difference between the python script and EFD values.}

\medskip }
\end{minipage} \\ \cdashline{2-2}

 & Actual Result \\
 & \begin{minipage}[t]{15cm}{\footnotesize

\medskip }
\end{minipage} \\ \cdashline{2-2}

 & Status: \textbf{ Not Executed } \\ \hline

\end{longtable}

\paragraph{Test Case LVV-T239 - M13T-021: Mirror Support Lifting and Parking }\mbox{}\\

Open  \href{https://jira.lsstcorp.org/secure/Tests.jspa#/testCase/LVV-T239}{\textit{ LVV-T239 } }
test case in Jira.

The M1M3 mirror support system must be able to ``lift up'' and ``park
down'' the M1M3 mirror both with automated control and in the
engineering mode. ~Lifting up is transitioning from passive support on
the statics supports to active support on the pneumatic figure control
actuators. ~Parking down is the opposite transition from active mirror
support to passive mirror support.

\textbf{ Preconditions}:\\
\begin{enumerate}
\tightlist
\item
  All M1M3 Systems installed, except thermal control and interlocks
\item
  All M1M3 Systems operating, except thermal control and interlocks
\item
  The accelerometers on the mirror are operational and the outputs are
  being recorded.
\item
  The independent measuring system (IMS) is operational and the outputs
  are being recorded.
\item
  The M1M3 mirror system is in active mode and the mirror is resting on
  the actuators
\item
  The M1M3 mirror system is in passive mode and the mirror is resting on
  the static supports.
\item
  The M1M3 mirror positioning system is operational and the outputs are
  being recorded.
\item
  Hard points are in their nominal positions with negligible loads
  present.
\item
  M1M3 control software is current and configured for this test.
\item
  All test data is being recorded to the Engineering Facility Database
  EFD
\end{enumerate}

Execution status: {\bf Not Executed }

Final comment:\\


Detailed steps results:

\begin{longtable}{p{1cm}p{15cm}}
\hline
{Step} & Step Details\\ \hline
1 & Description \\
 & \begin{minipage}[t]{15cm}
{\footnotesize
From a terminal run the following command after replacing yyyy with
year, mm with month, dd with day, hh with hour, and nn with
minute.\\[2\baselineskip]python36 M13T021.py \textbar{}\& tee
\textasciitilde{}/M13T021-yyyymmddhhnn.txt

\medskip }
\end{minipage}
\\ \cdashline{2-2}


 & Expected Result \\
 & \begin{minipage}[t]{15cm}{\footnotesize
N/A

\medskip }
\end{minipage} \\ \cdashline{2-2}

 & Actual Result \\
 & \begin{minipage}[t]{15cm}{\footnotesize

\medskip }
\end{minipage} \\ \cdashline{2-2}

 & Status: \textbf{ Not Executed } \\ \hline

2 & Description \\
 & \begin{minipage}[t]{15cm}
{\footnotesize
Wait for the script to complete.

\medskip }
\end{minipage}
\\ \cdashline{2-2}


 & Expected Result \\
 & \begin{minipage}[t]{15cm}{\footnotesize
N/A

\medskip }
\end{minipage} \\ \cdashline{2-2}

 & Actual Result \\
 & \begin{minipage}[t]{15cm}{\footnotesize

\medskip }
\end{minipage} \\ \cdashline{2-2}

 & Status: \textbf{ Not Executed } \\ \hline

3 & Description \\
 & \begin{minipage}[t]{15cm}
{\footnotesize
From a terminal run the following command replacing yyyy with year, mm
with month, dd with day, hh with hour, and nn with minute used in step
1.\\[2\baselineskip]grep -E ``ANOM -''
\textasciitilde{}/M13T021-yyyymmddhhnn.txt

\medskip }
\end{minipage}
\\ \cdashline{2-2}


 & Expected Result \\
 & \begin{minipage}[t]{15cm}{\footnotesize
The search should display no results meaning no anomalies occurred
during the test.

\medskip }
\end{minipage} \\ \cdashline{2-2}

 & Actual Result \\
 & \begin{minipage}[t]{15cm}{\footnotesize

\medskip }
\end{minipage} \\ \cdashline{2-2}

 & Status: \textbf{ Not Executed } \\ \hline

4 & Description \\
 & \begin{minipage}[t]{15cm}
{\footnotesize
Compress \textasciitilde{}/M13T021-yyyymmddhhnn.txt to
\textasciitilde{}/M13T021-yyyymmddhhnn.zip and attach it to this test
step.

\medskip }
\end{minipage}
\\ \cdashline{2-2}


 & Expected Result \\
 & \begin{minipage}[t]{15cm}{\footnotesize
N/A

\medskip }
\end{minipage} \\ \cdashline{2-2}

 & Actual Result \\
 & \begin{minipage}[t]{15cm}{\footnotesize

\medskip }
\end{minipage} \\ \cdashline{2-2}

 & Status: \textbf{ Not Executed } \\ \hline

\end{longtable}

\paragraph{Test Case LVV-T235 - M13T-012: Position Repeatability After Parking }\mbox{}\\

Open  \href{https://jira.lsstcorp.org/secure/Tests.jspa#/testCase/LVV-T235}{\textit{ LVV-T235 } }
test case in Jira.

Correct position of the M1M3 mirror relative to the camera and M2 mirror
is of paramount importance for maintaining image quality. When the
mirror support system transitions to the active mode it must be able to
repeatedly return to the desired location.

\textbf{ Preconditions}:\\
\begin{enumerate}
\tightlist
\item
  All M1M3 Systems installed, except thermal control and interlocks
\item
  All M1M3 Systems operating, except thermal control and interlocks
\item
  The accelerometers on the mirror are operational and the outputs are
  being recorded.
\item
  The independent measuring system (IMS) is operational and the outputs
  are being recorded.
\item
  The M1M3 mirror system is in active mode and the mirror is resting on
  the actuators
\item
  The M1M3 mirror positioning system is operational and the outputs are
  being recorded.
\item
  Hard points are in their nominal positions with negligible loads
  present.
\item
  M1M3 control software is current and configured for this test.
\item
  All test data is being recorded to the Engineering Facility Database
  EFD
\end{enumerate}

Execution status: {\bf Not Executed }

Final comment:\\


Detailed steps results:

\begin{longtable}{p{1cm}p{15cm}}
\hline
{Step} & Step Details\\ \hline
1 & Description \\
 & \begin{minipage}[t]{15cm}
{\footnotesize
From a terminal run the following command after replacing yyyy with
year, mm with month, dd with day, hh with hour, and nn with
minute.\\[2\baselineskip]python36 M13T012.py \textbar{}\& tee
\textasciitilde{}/M13T012-yyyymmddhhnn.txt

\medskip }
\end{minipage}
\\ \cdashline{2-2}


 & Expected Result \\
 & \begin{minipage}[t]{15cm}{\footnotesize
N/A

\medskip }
\end{minipage} \\ \cdashline{2-2}

 & Actual Result \\
 & \begin{minipage}[t]{15cm}{\footnotesize

\medskip }
\end{minipage} \\ \cdashline{2-2}

 & Status: \textbf{ Not Executed } \\ \hline

2 & Description \\
 & \begin{minipage}[t]{15cm}
{\footnotesize
Wait for the script to complete.

\medskip }
\end{minipage}
\\ \cdashline{2-2}


 & Expected Result \\
 & \begin{minipage}[t]{15cm}{\footnotesize
N/A

\medskip }
\end{minipage} \\ \cdashline{2-2}

 & Actual Result \\
 & \begin{minipage}[t]{15cm}{\footnotesize

\medskip }
\end{minipage} \\ \cdashline{2-2}

 & Status: \textbf{ Not Executed } \\ \hline

3 & Description \\
 & \begin{minipage}[t]{15cm}
{\footnotesize
From a terminal run the following command replacing yyyy with year, mm
with month, dd with day, hh with hour, and nn with minute used in step
1.\\[2\baselineskip]grep -E ``ANOM -''
\textasciitilde{}/M13T012-yyyymmddhhnn.txt

\medskip }
\end{minipage}
\\ \cdashline{2-2}


 & Expected Result \\
 & \begin{minipage}[t]{15cm}{\footnotesize
The search should display no results meaning no anomalies occurred
during the test.

\medskip }
\end{minipage} \\ \cdashline{2-2}

 & Actual Result \\
 & \begin{minipage}[t]{15cm}{\footnotesize

\medskip }
\end{minipage} \\ \cdashline{2-2}

 & Status: \textbf{ Not Executed } \\ \hline

4 & Description \\
 & \begin{minipage}[t]{15cm}
{\footnotesize
Compress \textasciitilde{}/M13T012-yyyymmddhhnn.txt to
\textasciitilde{}/M13T012-yyyymmddhhnn.zip and attach it to this test
step.

\medskip }
\end{minipage}
\\ \cdashline{2-2}


 & Expected Result \\
 & \begin{minipage}[t]{15cm}{\footnotesize
N/A

\medskip }
\end{minipage} \\ \cdashline{2-2}

 & Actual Result \\
 & \begin{minipage}[t]{15cm}{\footnotesize

\medskip }
\end{minipage} \\ \cdashline{2-2}

 & Status: \textbf{ Not Executed } \\ \hline

\end{longtable}

\paragraph{Test Case LVV-T234 - M13T-010: Position System Requirements }\mbox{}\\

Open  \href{https://jira.lsstcorp.org/secure/Tests.jspa#/testCase/LVV-T234}{\textit{ LVV-T234 } }
test case in Jira.

Positioning error in the M1M3 location produce misalignments in the
optical system between the M1M3, the M2 and the Camera. These
misalignments produce image degradation. The purpose of these
requirements is to verify that the M1M3 position can adequately
controlled.

\textbf{ Preconditions}:\\
\begin{enumerate}
\tightlist
\item
  All M1M3 Systems installed, except thermal control and interlocks
\item
  All M1M3 Systems operating, except thermal control and interlocks
\item
  The independent measuring system (IMS) is operational and the outputs
  are being recorded.
\item
  The M1M3 mirror system is in active mode and the mirror is resting on
  the actuators
\item
  The M1M3 mirror positioning system is operational and the outputs are
  being recorded.
\item
  Hard points are in their nominal positions with negligible loads
  present.
\item
  M1M3 control software is current and configured for this test.
\item
  All test data is being recorded to the Engineering Facility Database
  EFD
\end{enumerate}

Execution status: {\bf Not Executed }

Final comment:\\


Detailed steps results:

\begin{longtable}{p{1cm}p{15cm}}
\hline
{Step} & Step Details\\ \hline
1 & Description \\
 & \begin{minipage}[t]{15cm}
{\footnotesize
From a terminal run the following command after replacing yyyy with
year, mm with month, dd with day, hh with hour, and nn with
minute.\\[2\baselineskip]python36 M13T010.py \textbar{}\& tee
\textasciitilde{}/M13T010-yyyymmddhhnn.txt

\medskip }
\end{minipage}
\\ \cdashline{2-2}


 & Expected Result \\
 & \begin{minipage}[t]{15cm}{\footnotesize
N/A

\medskip }
\end{minipage} \\ \cdashline{2-2}

 & Actual Result \\
 & \begin{minipage}[t]{15cm}{\footnotesize

\medskip }
\end{minipage} \\ \cdashline{2-2}

 & Status: \textbf{ Not Executed } \\ \hline

2 & Description \\
 & \begin{minipage}[t]{15cm}
{\footnotesize
Wait for the script to complete.

\medskip }
\end{minipage}
\\ \cdashline{2-2}


 & Expected Result \\
 & \begin{minipage}[t]{15cm}{\footnotesize
N/A

\medskip }
\end{minipage} \\ \cdashline{2-2}

 & Actual Result \\
 & \begin{minipage}[t]{15cm}{\footnotesize

\medskip }
\end{minipage} \\ \cdashline{2-2}

 & Status: \textbf{ Not Executed } \\ \hline

3 & Description \\
 & \begin{minipage}[t]{15cm}
{\footnotesize
From a terminal run the following command replacing yyyy with year, mm
with month, dd with day, hh with hour, and nn with minute used in step
1.\\[2\baselineskip]grep -E ``ANOM -''
\textasciitilde{}/M13T010-yyyymmddhhnn.txt

\medskip }
\end{minipage}
\\ \cdashline{2-2}


 & Expected Result \\
 & \begin{minipage}[t]{15cm}{\footnotesize
The search should display no results meaning no anomalies occurred
during the test.

\medskip }
\end{minipage} \\ \cdashline{2-2}

 & Actual Result \\
 & \begin{minipage}[t]{15cm}{\footnotesize

\medskip }
\end{minipage} \\ \cdashline{2-2}

 & Status: \textbf{ Not Executed } \\ \hline

4 & Description \\
 & \begin{minipage}[t]{15cm}
{\footnotesize
Compress \textasciitilde{}/M13T010-yyyymmddhhnn.txt to
\textasciitilde{}/M13T010-yyyymmddhhnn.zip and attach it to this test
step.

\medskip }
\end{minipage}
\\ \cdashline{2-2}


 & Expected Result \\
 & \begin{minipage}[t]{15cm}{\footnotesize
N/A

\medskip }
\end{minipage} \\ \cdashline{2-2}

 & Actual Result \\
 & \begin{minipage}[t]{15cm}{\footnotesize

\medskip }
\end{minipage} \\ \cdashline{2-2}

 & Status: \textbf{ Not Executed } \\ \hline

\end{longtable}

\paragraph{Test Case LVV-T254 - M13T-013: Determination of X, Y, Z Zero Coordinate }\mbox{}\\

Open  \href{https://jira.lsstcorp.org/secure/Tests.jspa#/testCase/LVV-T254}{\textit{ LVV-T254 } }
test case in Jira.

The nominal operating position is in the center of the M1M3 operating
range. The operating range is limited by contact between mirror and the
static supports. This purpose of this test is to determine the nominal
operating range which is utilized throughout the testing process.
Consequently, this is one of the first tests that will need to be run.~

\textbf{ Preconditions}:\\
\begin{enumerate}
\tightlist
\item
  All M1M3 Systems installed, except thermal control and interlocks
\item
  All M1M3 Systems operating, except thermal control and interlocks
\item
  The independent measuring system (IMS) is operational and the outputs
  are being recorded.
\item
  The M1M3 mirror system is in active mode and the mirror is resting on
  the actuators
\item
  The M1M3 mirror positioning system is operational and the outputs are
  being recorded.
\item
  M1M3 control software is current and configured for this test.
\item
  All test data is being recorded to the Engineering Facility Database
  EFD
\end{enumerate}

Execution status: {\bf Not Executed }

Final comment:\\


Detailed steps results:

\begin{longtable}{p{1cm}p{15cm}}
\hline
{Step} & Step Details\\ \hline
1 & Description \\
 & \begin{minipage}[t]{15cm}
{\footnotesize
From a terminal run the following command after replacing yyyy with
year, mm with month, dd with day, hh with hour, and nn with
minute.\\[2\baselineskip]python36 M13T013.py \textbar{}\& tee
\textasciitilde{}/M13T013-yyyymmddhhnn.txt

\medskip }
\end{minipage}
\\ \cdashline{2-2}


 & Expected Result \\
 & \begin{minipage}[t]{15cm}{\footnotesize
N/A

\medskip }
\end{minipage} \\ \cdashline{2-2}

 & Actual Result \\
 & \begin{minipage}[t]{15cm}{\footnotesize

\medskip }
\end{minipage} \\ \cdashline{2-2}

 & Status: \textbf{ Not Executed } \\ \hline

2 & Description \\
 & \begin{minipage}[t]{15cm}
{\footnotesize
Wait for the script to complete.

\medskip }
\end{minipage}
\\ \cdashline{2-2}


 & Expected Result \\
 & \begin{minipage}[t]{15cm}{\footnotesize
N/A

\medskip }
\end{minipage} \\ \cdashline{2-2}

 & Actual Result \\
 & \begin{minipage}[t]{15cm}{\footnotesize

\medskip }
\end{minipage} \\ \cdashline{2-2}

 & Status: \textbf{ Not Executed } \\ \hline

3 & Description \\
 & \begin{minipage}[t]{15cm}
{\footnotesize
From a terminal run the following command replacing yyyy with year, mm
with month, dd with day, hh with hour, and nn with minute used in step
1.\\[2\baselineskip]grep -E ``ANOM -''
\textasciitilde{}/M13T013-yyyymmddhhnn.txt

\medskip }
\end{minipage}
\\ \cdashline{2-2}


 & Expected Result \\
 & \begin{minipage}[t]{15cm}{\footnotesize
The search should display no results meaning no anomalies occurred
during the test.

\medskip }
\end{minipage} \\ \cdashline{2-2}

 & Actual Result \\
 & \begin{minipage}[t]{15cm}{\footnotesize

\medskip }
\end{minipage} \\ \cdashline{2-2}

 & Status: \textbf{ Not Executed } \\ \hline

4 & Description \\
 & \begin{minipage}[t]{15cm}
{\footnotesize
Compress \textasciitilde{}/M13T013-yyyymmddhhnn.txt with its data files
to \textasciitilde{}/M13T013-yyyymmddhhnn.zip and attach it to this test
step.

\medskip }
\end{minipage}
\\ \cdashline{2-2}


 & Expected Result \\
 & \begin{minipage}[t]{15cm}{\footnotesize
N/A

\medskip }
\end{minipage} \\ \cdashline{2-2}

 & Actual Result \\
 & \begin{minipage}[t]{15cm}{\footnotesize

\medskip }
\end{minipage} \\ \cdashline{2-2}

 & Status: \textbf{ Not Executed } \\ \hline

5 & Description \\
 & \begin{minipage}[t]{15cm}
{\footnotesize
Review data to determine motion range and X, Y, Z, Rx, Ry, Rz coordinate
for the reference position.

\medskip }
\end{minipage}
\\ \cdashline{2-2}


 & Expected Result \\
 & \begin{minipage}[t]{15cm}{\footnotesize

\medskip }
\end{minipage} \\ \cdashline{2-2}

 & Actual Result \\
 & \begin{minipage}[t]{15cm}{\footnotesize

\medskip }
\end{minipage} \\ \cdashline{2-2}

 & Status: \textbf{ Not Executed } \\ \hline

6 & Description \\
 & \begin{minipage}[t]{15cm}
{\footnotesize
post results of data analysis here

\medskip }
\end{minipage}
\\ \cdashline{2-2}


 & Expected Result \\
 & \begin{minipage}[t]{15cm}{\footnotesize

\medskip }
\end{minipage} \\ \cdashline{2-2}

 & Actual Result \\
 & \begin{minipage}[t]{15cm}{\footnotesize

\medskip }
\end{minipage} \\ \cdashline{2-2}

 & Status: \textbf{ Not Executed } \\ \hline

\end{longtable}

\paragraph{Test Case LVV-T229 - M13T-002: Bump Test - parked engineering state }\mbox{}\\

Open  \href{https://jira.lsstcorp.org/secure/Tests.jspa#/testCase/LVV-T229}{\textit{ LVV-T229 } }
test case in Jira.

This test is intended to verify that all the actuators are correctly
installed and correctly communicating with the control system. It also
verifies that their support utilities (compressed air and power) are
correctly connected and operational.\texttt{}

\textbf{ Preconditions}:\\
- All m1m3 actuators installed and operational\\
- m1m3 control software ready\\
- m1m3 mirror system NOT in active mode\\
- m1m3 surrogate resting on static supports

Execution status: {\bf Not Executed }

Final comment:\\


Detailed steps results:

\begin{longtable}{p{1cm}p{15cm}}
\hline
{Step} & Step Details\\ \hline
1 & Description \\
 & \begin{minipage}[t]{15cm}
{\footnotesize
From a terminal run the following command after replacing yyyy with
year, mm with month, dd with day, hh with hour, and nn with
minute.\\[2\baselineskip]python36 M13T002.py \textbar{}\& tee
\textasciitilde{}/M13T002-yyyymmddhhnn.txt

\medskip }
\end{minipage}
\\ \cdashline{2-2}


 & Expected Result \\
 & \begin{minipage}[t]{15cm}{\footnotesize
N/A

\medskip }
\end{minipage} \\ \cdashline{2-2}

 & Actual Result \\
 & \begin{minipage}[t]{15cm}{\footnotesize

\medskip }
\end{minipage} \\ \cdashline{2-2}

 & Status: \textbf{ Not Executed } \\ \hline

2 & Description \\
 & \begin{minipage}[t]{15cm}
{\footnotesize
Wait for the script to complete.

\medskip }
\end{minipage}
\\ \cdashline{2-2}


 & Expected Result \\
 & \begin{minipage}[t]{15cm}{\footnotesize
N/A

\medskip }
\end{minipage} \\ \cdashline{2-2}

 & Actual Result \\
 & \begin{minipage}[t]{15cm}{\footnotesize

\medskip }
\end{minipage} \\ \cdashline{2-2}

 & Status: \textbf{ Not Executed } \\ \hline

3 & Description \\
 & \begin{minipage}[t]{15cm}
{\footnotesize
From a terminal run the following command replacing yyyy with year, mm
with month, dd with day, hh with hour, and nn with minute used in step
1.\\[2\baselineskip]grep -E ``ANOM -''
\textasciitilde{}/M13T002-yyyymmddhhnn.txt

\medskip }
\end{minipage}
\\ \cdashline{2-2}


 & Expected Result \\
 & \begin{minipage}[t]{15cm}{\footnotesize
The search should display no results meaning no anomalies occurred
during the test.

\medskip }
\end{minipage} \\ \cdashline{2-2}

 & Actual Result \\
 & \begin{minipage}[t]{15cm}{\footnotesize

\medskip }
\end{minipage} \\ \cdashline{2-2}

 & Status: \textbf{ Not Executed } \\ \hline

4 & Description \\
 & \begin{minipage}[t]{15cm}
{\footnotesize
Compress \textasciitilde{}/M13T002-yyyymmddhhnn.txt to
\textasciitilde{}/M13T002-yyyymmddhhnn.zip and attach it to this test
step.

\medskip }
\end{minipage}
\\ \cdashline{2-2}


 & Expected Result \\
 & \begin{minipage}[t]{15cm}{\footnotesize
N/A

\medskip }
\end{minipage} \\ \cdashline{2-2}

 & Actual Result \\
 & \begin{minipage}[t]{15cm}{\footnotesize

\medskip }
\end{minipage} \\ \cdashline{2-2}

 & Status: \textbf{ Not Executed } \\ \hline

\end{longtable}

\paragraph{Test Case LVV-T238 - M13T-018: Automatic Mode (bump test in active engineering state) }\mbox{}\\

Open  \href{https://jira.lsstcorp.org/secure/Tests.jspa#/testCase/LVV-T238}{\textit{ LVV-T238 } }
test case in Jira.

Effectively demostrating the operation of
M1M3\\[2\baselineskip]\texttt{\#\ Steps:\#\ -\ Transition\ from\ standby\ to\ active\ engineering\ state\#\ -\ Perform\ the\ following\ steps\ for\ each\ force\ actuator\#\ ~\ -\ If\ the\ force\ actuator\ has\ an\ X\ component\#\ ~\ ~\ -\ Apply\ a\ pure\ X\ force\ offset\#\ ~\ ~\ -\ Verify\ the\ pure\ X\ force\ is\ being\ applied\#\ ~\ ~\ -\ Verify\ the\ pure\ X\ force\ is\ being\ measured\#\ ~\ ~\ -\ Clear\ offset\ forces\#\ ~\ ~\ -\ Verify\ the\ pure\ X\ force\ is\ no\ longer\ being\ applied\#\ ~\ ~\ -\ Verify\ the\ pure\ X\ force\ is\ no\ longer\ being\ measured\#\ ~\ ~\ -\ Apply\ a\ pure\ -X\ force\ offset\#\ ~\ ~\ -\ Verify\ the\ pure\ -X\ force\ is\ being\ applied\#\ ~\ ~\ -\ Verify\ the\ pure\ -X\ force\ is\ being\ measured\#\ ~\ ~\ -\ Clear\ offset\ forces\#\ ~\ ~\ -\ Verify\ the\ pure\ -X\ force\ is\ no\ longer\ being\ applied\#\ ~\ ~\ -\ Verify\ the\ pure\ -X\ force\ is\ no\ longer\ being\ measured\#\ ~\ -\ If\ the\ force\ actuator\ has\ an\ Y\ component\#\ ~\ ~\ -\ Apply\ a\ pure\ Y\ force\ offset\#\ ~\ ~\ -\ Verify\ the\ pure\ Y\ force\ is\ being\ applied\#\ ~\ ~\ -\ Verify\ the\ pure\ Y\ force\ is\ being\ measured\#\ ~\ ~\ -\ Clear\ offset\ forces\#\ ~\ ~\ -\ Verify\ the\ pure\ Y\ force\ is\ no\ longer\ being\ applied\#\ ~\ ~\ -\ Verify\ the\ pure\ Y\ force\ is\ no\ longer\ being\ measured\#\ ~\ ~\ -\ Apply\ a\ pure\ -Y\ force\ offset\#\ ~\ ~\ -\ Verify\ the\ pure\ -Y\ force\ is\ being\ applied\#\ ~\ ~\ -\ Verify\ the\ pure\ -Y\ force\ is\ being\ measured\#\ ~\ ~\ -\ Clear\ offset\ forces\#\ ~\ ~\ -\ Verify\ the\ pure\ -Y\ force\ is\ no\ longer\ being\ applied\#\ ~\ ~\ -\ Verify\ the\ pure\ -Y\ force\ is\ no\ longer\ being\ measured\#\ ~\ -\ Apply\ a\ pure\ Z\ force\ offset\#\ ~\ -\ Verify\ the\ pure\ Z\ force\ is\ being\ applied\#\ ~\ -\ Verify\ the\ pure\ Z\ force\ is\ being\ measured\#\ ~\ -\ Clear\ offset\ forces\#\ ~\ -\ Verify\ the\ pure\ Z\ force\ is\ no\ longer\ being\ applied\#\ ~\ -\ Verify\ the\ pure\ Z\ force\ is\ no\ longer\ being\ measured\#\ ~\ -\ Apply\ a\ pure\ -Z\ force\ offset\#\ ~\ -\ Verify\ the\ pure\ -Z\ force\ is\ being\ applied\#\ ~\ -\ Verify\ the\ pure\ -Z\ force\ is\ being\ measured\#\ ~\ -\ Clear\ offset\ forces\#\ ~\ -\ Verify\ the\ pure\ -Z\ force\ is\ no\ longer\ being\ applied\#\ ~\ -\ Verify\ the\ pure\ -Z\ force\ is\ no\ longer\ being\ measured\#\ -\ Transition\ from\ active\ engineering\ state\ to\ standby}

\textbf{ Preconditions}:\\
\begin{longtable}[]{@{}l@{}}
\toprule
\begin{minipage}[t]{0.97\columnwidth}\raggedright\strut
\begin{enumerate}
\tightlist
\item
  All M1M3 Systems installed, except thermal control and interlocks
\item
  All M1M3 Systems operating, except thermal control and interlocks
\item
  The accelerometers on the mirror are operational and the outputs are
  being recorded.
\item
  The independent measuring system (IMS) is operational and the outputs
  are being recorded.
\item
  The M1M3 mirror system is in active mode and the mirror is resting on
  the actuators
\item
  The M1M3 mirror system is in passive mode and the mirror is resting on
  the static supports.
\item
  The M1M3 mirror positioning system is operational and the outputs are
  being recorded.
\item
  Hard points are in their nominal positions with negligible loads
  present.
\item
  M1M3 control software is current and configured for this test.
\item
  All test data is being recorded to the Engineering Facility Database
  EFD
\end{enumerate}\strut
\end{minipage}\tabularnewline
\bottomrule
\end{longtable}

Execution status: {\bf Not Executed }

Final comment:\\


Detailed steps results:

\begin{longtable}{p{1cm}p{15cm}}
\hline
{Step} & Step Details\\ \hline
1 & Description \\
 & \begin{minipage}[t]{15cm}
{\footnotesize
From a terminal run the following command after replacing yyyy with
year, mm with month, dd with day, hh with hour, and nn with
minute.\\[2\baselineskip]python36 M13T018.py \textbar{}\& tee
\textasciitilde{}/M13T018-yyyymmddhhnn.txt

\medskip }
\end{minipage}
\\ \cdashline{2-2}


 & Expected Result \\
 & \begin{minipage}[t]{15cm}{\footnotesize
N/A

\medskip }
\end{minipage} \\ \cdashline{2-2}

 & Actual Result \\
 & \begin{minipage}[t]{15cm}{\footnotesize

\medskip }
\end{minipage} \\ \cdashline{2-2}

 & Status: \textbf{ Not Executed } \\ \hline

2 & Description \\
 & \begin{minipage}[t]{15cm}
{\footnotesize
Wait for the script to complete.

\medskip }
\end{minipage}
\\ \cdashline{2-2}


 & Expected Result \\
 & \begin{minipage}[t]{15cm}{\footnotesize
N/A

\medskip }
\end{minipage} \\ \cdashline{2-2}

 & Actual Result \\
 & \begin{minipage}[t]{15cm}{\footnotesize

\medskip }
\end{minipage} \\ \cdashline{2-2}

 & Status: \textbf{ Not Executed } \\ \hline

3 & Description \\
 & \begin{minipage}[t]{15cm}
{\footnotesize
From a terminal run the following command replacing yyyy with year, mm
with month, dd with day, hh with hour, and nn with minute used in step
1.\\[2\baselineskip]grep -E ``ANOM -''
\textasciitilde{}/M13T018-yyyymmddhhnn.txt

\medskip }
\end{minipage}
\\ \cdashline{2-2}


 & Expected Result \\
 & \begin{minipage}[t]{15cm}{\footnotesize
The search should display no results meaning no anomalies occurred
during the test.

\medskip }
\end{minipage} \\ \cdashline{2-2}

 & Actual Result \\
 & \begin{minipage}[t]{15cm}{\footnotesize

\medskip }
\end{minipage} \\ \cdashline{2-2}

 & Status: \textbf{ Not Executed } \\ \hline

4 & Description \\
 & \begin{minipage}[t]{15cm}
{\footnotesize
Compress \textasciitilde{}/M13T018-yyyymmddhhnn.txt along with its
associated data files in the home directory to
\textasciitilde{}/M13T018-yyyymmddhhnn.zip and attach it to this test
step.

\medskip }
\end{minipage}
\\ \cdashline{2-2}


 & Expected Result \\
 & \begin{minipage}[t]{15cm}{\footnotesize
N/A

\medskip }
\end{minipage} \\ \cdashline{2-2}

 & Actual Result \\
 & \begin{minipage}[t]{15cm}{\footnotesize

\medskip }
\end{minipage} \\ \cdashline{2-2}

 & Status: \textbf{ Not Executed } \\ \hline

\end{longtable}

\paragraph{Test Case LVV-T230 - M13T-003: Individual Hardpoint Displacement Test }\mbox{}\\

Open  \href{https://jira.lsstcorp.org/secure/Tests.jspa#/testCase/LVV-T230}{\textit{ LVV-T230 } }
test case in Jira.

This test is intended to verify that all the hardpoints are correctly
installed and correctly communicating with the control system. It also
verifies that their support utilities (power) are correctly connected
and operational.\\[2\baselineskip]

\begin{verbatim}
# Steps:
# - Transition from standby to parked engineering state
# - Perform the following steps for each hardpoint actuator
#   - Perform the following steps for full extension and full retraction
#     - Issue hardpoint step command
#     - Verify hardpoint is moving
#     - Wait for hardpoint motion to complete or a limit switch is operated
#     - Issue stop hardpoint motion command
#     - Verify hardpoint is stopped
#     - Query EFD for hardpoint monitor data for test duration
#     - Query EFD for hardpoint actuator data for test duration
# - Transition from parked engineering to standby state
\end{verbatim}

\textbf{ Preconditions}:\\
\begin{enumerate}
\tightlist
\item
  All M1M3 hardpoints installed and operational, however, the hardpoint
  pneumatic breakaway mechanism are not energized (no compressed air to
  the breakaways)
\item
  M1M3 limit switches have been installed properly and the
  directionality has been tested at the lab before integration into the
  mirror cell.
\item
  M1M3 control software is current and configured for this test.
\item
  The M1M3 mirror system is in NOT in active mode, and the surrogate
  mirror is resting on the static supports.
\end{enumerate}

Execution status: {\bf Not Executed }

Final comment:\\


Detailed steps results:

\begin{longtable}{p{1cm}p{15cm}}
\hline
{Step} & Step Details\\ \hline
1 & Description \\
 & \begin{minipage}[t]{15cm}
{\footnotesize
Close the air valve inside the mirror cell and wait 5 minutes for the
air in the cell to vent.

\medskip }
\end{minipage}
\\ \cdashline{2-2}


 & Expected Result \\
 & \begin{minipage}[t]{15cm}{\footnotesize
N/A

\medskip }
\end{minipage} \\ \cdashline{2-2}

 & Actual Result \\
 & \begin{minipage}[t]{15cm}{\footnotesize

\medskip }
\end{minipage} \\ \cdashline{2-2}

 & Status: \textbf{ Not Executed } \\ \hline

2 & Description \\
 & \begin{minipage}[t]{15cm}
{\footnotesize
From a terminal run the following command after replacing yyyy with
year, mm with month, dd with day, hh with hour, and nn with
minute.\\[2\baselineskip]python36 M13T003.py \textbar{}\& tee
\textasciitilde{}/M13T003-yyyymmddhhnn.txt

\medskip }
\end{minipage}
\\ \cdashline{2-2}


 & Expected Result \\
 & \begin{minipage}[t]{15cm}{\footnotesize
N/A

\medskip }
\end{minipage} \\ \cdashline{2-2}

 & Actual Result \\
 & \begin{minipage}[t]{15cm}{\footnotesize

\medskip }
\end{minipage} \\ \cdashline{2-2}

 & Status: \textbf{ Not Executed } \\ \hline

3 & Description \\
 & \begin{minipage}[t]{15cm}
{\footnotesize
Wait for the script to complete.

\medskip }
\end{minipage}
\\ \cdashline{2-2}


 & Expected Result \\
 & \begin{minipage}[t]{15cm}{\footnotesize
N/A

\medskip }
\end{minipage} \\ \cdashline{2-2}

 & Actual Result \\
 & \begin{minipage}[t]{15cm}{\footnotesize

\medskip }
\end{minipage} \\ \cdashline{2-2}

 & Status: \textbf{ Not Executed } \\ \hline

4 & Description \\
 & \begin{minipage}[t]{15cm}
{\footnotesize
From a terminal run the following command replacing yyyy with year, mm
with month, dd with day, hh with hour, and nn with minute used in step
1.\\[2\baselineskip]grep -E ``ANOM -''
\textasciitilde{}/M13T003-yyyymmddhhnn.txt

\medskip }
\end{minipage}
\\ \cdashline{2-2}


 & Expected Result \\
 & \begin{minipage}[t]{15cm}{\footnotesize
The search should display no results meaning no anomalies occurred
during the test.

\medskip }
\end{minipage} \\ \cdashline{2-2}

 & Actual Result \\
 & \begin{minipage}[t]{15cm}{\footnotesize

\medskip }
\end{minipage} \\ \cdashline{2-2}

 & Status: \textbf{ Not Executed } \\ \hline

5 & Description \\
 & \begin{minipage}[t]{15cm}
{\footnotesize
Compress \textasciitilde{}/M13T003-yyyymmddhhnn.txt along with its
associated data files in the home directory to
\textasciitilde{}/M13T003-yyyymmddhhnn.zip and attach it to this test
step.

\medskip }
\end{minipage}
\\ \cdashline{2-2}


 & Expected Result \\
 & \begin{minipage}[t]{15cm}{\footnotesize
N/A

\medskip }
\end{minipage} \\ \cdashline{2-2}

 & Actual Result \\
 & \begin{minipage}[t]{15cm}{\footnotesize

\medskip }
\end{minipage} \\ \cdashline{2-2}

 & Status: \textbf{ Not Executed } \\ \hline

6 & Description \\
 & \begin{minipage}[t]{15cm}
{\footnotesize
Open the air valve inside the mirror cell.

\medskip }
\end{minipage}
\\ \cdashline{2-2}


 & Expected Result \\
 & \begin{minipage}[t]{15cm}{\footnotesize
N/A

\medskip }
\end{minipage} \\ \cdashline{2-2}

 & Actual Result \\
 & \begin{minipage}[t]{15cm}{\footnotesize

\medskip }
\end{minipage} \\ \cdashline{2-2}

 & Status: \textbf{ Not Executed } \\ \hline

\end{longtable}

\paragraph{Test Case LVV-T231 - M13T-004: Individual Hardpoint Breakaway Test }\mbox{}\\

Open  \href{https://jira.lsstcorp.org/secure/Tests.jspa#/testCase/LVV-T231}{\textit{ LVV-T231 } }
test case in Jira.

This test is intended to verify that all the hardpoints breakaway
mechanisms are correctly installed and the corresponding displacement
sensor is capable of detecting a breakaway control system. It also
verifies that their support utilities (power and compressed air) are
correctly connected and operational.\\
This test is effectively identical to the breakaway testing first
conducted in the lab. The purpose of this test is to verify that the
breakaway mechanisms are still functioning correctly after
installation\\[2\baselineskip]

\begin{verbatim}
# Steps:
# - Transition from standby to parked engineering state
# - Perform the following steps for each hardpoint actuator
#   - Perform the following steps for full extension and full retraction
#     - Issue hardpoint step command
#     - Verify hardpoint is moving
#     - Wait for hardpoint motion to complete or a limit switch is operated
#     - Issue stop hardpoint motion command
#     - Verify hardpoint is stopped
#     - Query EFD for hardpoint monitor data for test duration
#     - Query EFD for hardpoint actuator data for test duration
# - Transition from parked engineering to standby state
\end{verbatim}

\textbf{ Preconditions}:\\
\begin{enumerate}
\tightlist
\item
  All M1M3 hardpoints installed and operational
\item
  All hardpoints limit switches have been tested.
\item
  M1M3 control software is current and configured for this test.
\item
  The M1M3 mirror system is in NOT in active mode, and the surrogate
  mirror is resting on the static supports.
\item
  The hardpoint breakaway mechanisms are energized and supplied with
  compressed air
\end{enumerate}

Execution status: {\bf Not Executed }

Final comment:\\


Detailed steps results:

\begin{longtable}{p{1cm}p{15cm}}
\hline
{Step} & Step Details\\ \hline
1 & Description \\
 & \begin{minipage}[t]{15cm}
{\footnotesize
From a terminal run the following command after replacing yyyy with
year, mm with month, dd with day, hh with hour, and nn with
minute.\\[2\baselineskip]python36 M13T004.py \textbar{}\& tee
\textasciitilde{}/M13T004-yyyymmddhhnn.txt

\medskip }
\end{minipage}
\\ \cdashline{2-2}


 & Expected Result \\
 & \begin{minipage}[t]{15cm}{\footnotesize
N/A

\medskip }
\end{minipage} \\ \cdashline{2-2}

 & Actual Result \\
 & \begin{minipage}[t]{15cm}{\footnotesize

\medskip }
\end{minipage} \\ \cdashline{2-2}

 & Status: \textbf{ Not Executed } \\ \hline

2 & Description \\
 & \begin{minipage}[t]{15cm}
{\footnotesize
Wait for the script to complete.

\medskip }
\end{minipage}
\\ \cdashline{2-2}


 & Expected Result \\
 & \begin{minipage}[t]{15cm}{\footnotesize
N/A

\medskip }
\end{minipage} \\ \cdashline{2-2}

 & Actual Result \\
 & \begin{minipage}[t]{15cm}{\footnotesize

\medskip }
\end{minipage} \\ \cdashline{2-2}

 & Status: \textbf{ Not Executed } \\ \hline

3 & Description \\
 & \begin{minipage}[t]{15cm}
{\footnotesize
From a terminal run the following command replacing yyyy with year, mm
with month, dd with day, hh with hour, and nn with minute used in step
1.\\[2\baselineskip]grep -E ``ANOM -''
\textasciitilde{}/M13T003-yyyymmddhhnn.txt

\medskip }
\end{minipage}
\\ \cdashline{2-2}


 & Expected Result \\
 & \begin{minipage}[t]{15cm}{\footnotesize
N/A

\medskip }
\end{minipage} \\ \cdashline{2-2}

 & Actual Result \\
 & \begin{minipage}[t]{15cm}{\footnotesize

\medskip }
\end{minipage} \\ \cdashline{2-2}

 & Status: \textbf{ Not Executed } \\ \hline

4 & Description \\
 & \begin{minipage}[t]{15cm}
{\footnotesize
From a terminal run the following command replacing yyyy with year, mm
with month, dd with day, hh with hour, and nn with minute used in step
1.\\[2\baselineskip]grep -E ``ANOM -''
\textasciitilde{}/M13T004-yyyymmddhhnn.txt

\medskip }
\end{minipage}
\\ \cdashline{2-2}


 & Expected Result \\
 & \begin{minipage}[t]{15cm}{\footnotesize
The search should display no results meaning no anomalies occurred
during the test.

\medskip }
\end{minipage} \\ \cdashline{2-2}

 & Actual Result \\
 & \begin{minipage}[t]{15cm}{\footnotesize

\medskip }
\end{minipage} \\ \cdashline{2-2}

 & Status: \textbf{ Not Executed } \\ \hline

5 & Description \\
 & \begin{minipage}[t]{15cm}
{\footnotesize
Compress \textasciitilde{}/M13T004-yyyymmddhhnn.txt along with its
associated data files in the home directory to
\textasciitilde{}/M13T004-yyyymmddhhnn.zip and attach it to this test
step.

\medskip }
\end{minipage}
\\ \cdashline{2-2}


 & Expected Result \\
 & \begin{minipage}[t]{15cm}{\footnotesize
N/A

\medskip }
\end{minipage} \\ \cdashline{2-2}

 & Actual Result \\
 & \begin{minipage}[t]{15cm}{\footnotesize

\medskip }
\end{minipage} \\ \cdashline{2-2}

 & Status: \textbf{ Not Executed } \\ \hline

\end{longtable}

\paragraph{Test Case LVV-T232 - M13T-005: Hardpoint Force Balance System }\mbox{}\\

Open  \href{https://jira.lsstcorp.org/secure/Tests.jspa#/testCase/LVV-T232}{\textit{ LVV-T232 } }
test case in Jira.

This test is intended to verify that all the hardpoints force balance
system is correctly operating.\\
This system reads the forces, determined on their load cells, for each
of the 6 hardpoints.\\
The corresponding net forces and moments in the X, Y and Z ~directions
are determined. Distributed forces are then sent to the individual
actuators to counteract these net forces.~

\textbf{ Preconditions}:\\
\begin{enumerate}
\tightlist
\item
  Entire M1M3 mirror support systems, except thermal control and
  interlocks, are installed and operational.
\item
  M1M3 control software is current and configured for this test.
\item
  The M1M3 mirror system is in active mode and the mirror is resting on
  the actuators
\item
  The M1M3 surrogate mirror is located by the hardpoints near its
  nominal operating position.
\item
  Hard points are supporting negligible loads.
\item
  The M1M3 mirror positioning system is operational and the outputs are
  being recorded.
\item
  The accelerometers on the mirror are operational and the outputs are
  being recorded.
\item
  The independent measuring system (IMS) is operational and the outputs
  are being recorded.
\item
  All test data is being recorded to the Engineering Facility Database
  EFD
\end{enumerate}

Execution status: {\bf Not Executed }

Final comment:\\


Detailed steps results:

\begin{longtable}{p{1cm}p{15cm}}
\hline
{Step} & Step Details\\ \hline
1 & Description \\
 & \begin{minipage}[t]{15cm}
{\footnotesize
In the EUI click the Start button.

\medskip }
\end{minipage}
\\ \cdashline{2-2}


 & Expected Result \\
 & \begin{minipage}[t]{15cm}{\footnotesize
N/A

\medskip }
\end{minipage} \\ \cdashline{2-2}

 & Actual Result \\
 & \begin{minipage}[t]{15cm}{\footnotesize

\medskip }
\end{minipage} \\ \cdashline{2-2}

 & Status: \textbf{ Not Executed } \\ \hline

2 & Description \\
 & \begin{minipage}[t]{15cm}
{\footnotesize
In the EUI click the Enable button.

\medskip }
\end{minipage}
\\ \cdashline{2-2}


 & Expected Result \\
 & \begin{minipage}[t]{15cm}{\footnotesize
N/A

\medskip }
\end{minipage} \\ \cdashline{2-2}

 & Actual Result \\
 & \begin{minipage}[t]{15cm}{\footnotesize

\medskip }
\end{minipage} \\ \cdashline{2-2}

 & Status: \textbf{ Not Executed } \\ \hline

3 & Description \\
 & \begin{minipage}[t]{15cm}
{\footnotesize
In the EUI click the Enter Engineering button.

\medskip }
\end{minipage}
\\ \cdashline{2-2}


 & Expected Result \\
 & \begin{minipage}[t]{15cm}{\footnotesize
N/A

\medskip }
\end{minipage} \\ \cdashline{2-2}

 & Actual Result \\
 & \begin{minipage}[t]{15cm}{\footnotesize

\medskip }
\end{minipage} \\ \cdashline{2-2}

 & Status: \textbf{ Not Executed } \\ \hline

4 & Description \\
 & \begin{minipage}[t]{15cm}
{\footnotesize
In the EUI click the Raise Mirror button.

\medskip }
\end{minipage}
\\ \cdashline{2-2}


 & Expected Result \\
 & \begin{minipage}[t]{15cm}{\footnotesize
N/A

\medskip }
\end{minipage} \\ \cdashline{2-2}

 & Actual Result \\
 & \begin{minipage}[t]{15cm}{\footnotesize

\medskip }
\end{minipage} \\ \cdashline{2-2}

 & Status: \textbf{ Not Executed } \\ \hline

5 & Description \\
 & \begin{minipage}[t]{15cm}
{\footnotesize
Wait until the detailed state reports ActiveEngineeringState.

\medskip }
\end{minipage}
\\ \cdashline{2-2}


 & Expected Result \\
 & \begin{minipage}[t]{15cm}{\footnotesize
Detailed State is ActiveEngineeringState

\medskip }
\end{minipage} \\ \cdashline{2-2}

 & Actual Result \\
 & \begin{minipage}[t]{15cm}{\footnotesize

\medskip }
\end{minipage} \\ \cdashline{2-2}

 & Status: \textbf{ Not Executed } \\ \hline

6 & Description \\
 & \begin{minipage}[t]{15cm}
{\footnotesize
In the EUI navigate to the Hardpoints tab and click the Disable
Hardpoint Corrections Process button.

\medskip }
\end{minipage}
\\ \cdashline{2-2}


 & Expected Result \\
 & \begin{minipage}[t]{15cm}{\footnotesize
N/A

\medskip }
\end{minipage} \\ \cdashline{2-2}

 & Actual Result \\
 & \begin{minipage}[t]{15cm}{\footnotesize

\medskip }
\end{minipage} \\ \cdashline{2-2}

 & Status: \textbf{ Not Executed } \\ \hline

7 & Description \\
 & \begin{minipage}[t]{15cm}
{\footnotesize
Collect a couple of seconds of following types:

\begin{enumerate}
\tightlist
\item
  Baseline
\item
  20lbs +X
\item
  20lbs -X
\item
  20lbs +Y
\item
  20lbs -Y
\item
  20lbs +Z
\item
  20lbs -Z
\item
  20lbs +MX
\item
  20lbs -MX
\item
  20lbs +MY
\item
  20lbs -MY
\item
  20lbs +MZ
\item
  20lbs -MZ
\end{enumerate}

For each name the data file M13T005-Measured-\textbf{TYPE}-yyyymmdd.csv

\medskip }
\end{minipage}
\\ \cdashline{2-2}


 & Expected Result \\
 & \begin{minipage}[t]{15cm}{\footnotesize
N/A

\medskip }
\end{minipage} \\ \cdashline{2-2}

 & Actual Result \\
 & \begin{minipage}[t]{15cm}{\footnotesize

\medskip }
\end{minipage} \\ \cdashline{2-2}

 & Status: \textbf{ Not Executed } \\ \hline

8 & Description \\
 & \begin{minipage}[t]{15cm}
{\footnotesize
In the EUI navigate to the Hardpoints tab and click the Enable Hardpoint
Corrections Prcoess button.

\medskip }
\end{minipage}
\\ \cdashline{2-2}


 & Expected Result \\
 & \begin{minipage}[t]{15cm}{\footnotesize
N/A

\medskip }
\end{minipage} \\ \cdashline{2-2}

 & Actual Result \\
 & \begin{minipage}[t]{15cm}{\footnotesize

\medskip }
\end{minipage} \\ \cdashline{2-2}

 & Status: \textbf{ Not Executed } \\ \hline

9 & Description \\
 & \begin{minipage}[t]{15cm}
{\footnotesize
Collect a couple of seconds of following types:

\begin{enumerate}
\tightlist
\item
  Baseline
\item
  20lbs +X
\item
  20lbs -X
\item
  20lbs +Y
\item
  20lbs -Y
\item
  20lbs +Z
\item
  20lbs -Z
\item
  20lbs +MX
\item
  20lbs -MX
\item
  20lbs +MY
\item
  20lbs -MY
\item
  20lbs +MZ
\item
  20lbs -MZ
\end{enumerate}

For each name the data file M13T005-Corrected-\textbf{TYPE}-yyyymmdd.csv

\medskip }
\end{minipage}
\\ \cdashline{2-2}


 & Expected Result \\
 & \begin{minipage}[t]{15cm}{\footnotesize
N/A

\medskip }
\end{minipage} \\ \cdashline{2-2}

 & Actual Result \\
 & \begin{minipage}[t]{15cm}{\footnotesize

\medskip }
\end{minipage} \\ \cdashline{2-2}

 & Status: \textbf{ Not Executed } \\ \hline

10 & Description \\
 & \begin{minipage}[t]{15cm}
{\footnotesize
In the EUI click the Lower Mirror button.

\medskip }
\end{minipage}
\\ \cdashline{2-2}


 & Expected Result \\
 & \begin{minipage}[t]{15cm}{\footnotesize
N/A

\medskip }
\end{minipage} \\ \cdashline{2-2}

 & Actual Result \\
 & \begin{minipage}[t]{15cm}{\footnotesize

\medskip }
\end{minipage} \\ \cdashline{2-2}

 & Status: \textbf{ Not Executed } \\ \hline

11 & Description \\
 & \begin{minipage}[t]{15cm}
{\footnotesize
In the EUI click the Disable button.

\medskip }
\end{minipage}
\\ \cdashline{2-2}


 & Expected Result \\
 & \begin{minipage}[t]{15cm}{\footnotesize
N/A

\medskip }
\end{minipage} \\ \cdashline{2-2}

 & Actual Result \\
 & \begin{minipage}[t]{15cm}{\footnotesize

\medskip }
\end{minipage} \\ \cdashline{2-2}

 & Status: \textbf{ Not Executed } \\ \hline

12 & Description \\
 & \begin{minipage}[t]{15cm}
{\footnotesize
In the EUI click the Standby button.

\medskip }
\end{minipage}
\\ \cdashline{2-2}


 & Expected Result \\
 & \begin{minipage}[t]{15cm}{\footnotesize
N/A

\medskip }
\end{minipage} \\ \cdashline{2-2}

 & Actual Result \\
 & \begin{minipage}[t]{15cm}{\footnotesize

\medskip }
\end{minipage} \\ \cdashline{2-2}

 & Status: \textbf{ Not Executed } \\ \hline

13 & Description \\
 & \begin{minipage}[t]{15cm}
{\footnotesize
Compress \textasciitilde{}/M13T005-*.csv to
\textasciitilde{}/M13T005-yyyymmddhhnn.zip and attach it to this test
step.

\medskip }
\end{minipage}
\\ \cdashline{2-2}


 & Expected Result \\
 & \begin{minipage}[t]{15cm}{\footnotesize
N/A

\medskip }
\end{minipage} \\ \cdashline{2-2}

 & Actual Result \\
 & \begin{minipage}[t]{15cm}{\footnotesize

\medskip }
\end{minipage} \\ \cdashline{2-2}

 & Status: \textbf{ Not Executed } \\ \hline

14 & Description \\
 & \begin{minipage}[t]{15cm}
{\footnotesize
Process the data files in M13T005-yyyymmddhhnn.zip and determine the
accuracy of the measurements on Fx, Fy, Fz, Mx, My, Mz and post the
results here.

\medskip }
\end{minipage}
\\ \cdashline{2-2}


 & Expected Result \\
 & \begin{minipage}[t]{15cm}{\footnotesize

\medskip }
\end{minipage} \\ \cdashline{2-2}

 & Actual Result \\
 & \begin{minipage}[t]{15cm}{\footnotesize

\medskip }
\end{minipage} \\ \cdashline{2-2}

 & Status: \textbf{ Not Executed } \\ \hline

15 & Description \\
 & \begin{minipage}[t]{15cm}
{\footnotesize
Process the data files in M13T005-yyyymmddhhnn.zip and determine how
well the force balance system corrected Fx, Fy, Fz, Mx, My, Mz and post
the results here.

\medskip }
\end{minipage}
\\ \cdashline{2-2}


 & Expected Result \\
 & \begin{minipage}[t]{15cm}{\footnotesize

\medskip }
\end{minipage} \\ \cdashline{2-2}

 & Actual Result \\
 & \begin{minipage}[t]{15cm}{\footnotesize

\medskip }
\end{minipage} \\ \cdashline{2-2}

 & Status: \textbf{ Not Executed } \\ \hline

\end{longtable}

\paragraph{Test Case LVV-T241 - M13T-027: Actuator Force Limits }\mbox{}\\

Open  \href{https://jira.lsstcorp.org/secure/Tests.jspa#/testCase/LVV-T241}{\textit{ LVV-T241 } }
test case in Jira.

The safe force limit for every actuator was determined through analysis.
Mirror safety mandates that the mirror control system not send force
commanded to the actuators that exceed these limits.\\[2\baselineskip]

\begin{verbatim}
# Steps:
# - Transition from standby to parked engineering state
# - Perform the following steps for each force actuator
#   - If the force actuator has an X component
#     - Apply a X force offset 15% higher than the max X limit
#     - Verify the X force is rejected
#     - Verify the limited X force is applied
#     - Verify the limited X force is being measured
#     - Apply a X force offset 15% lower than the min X limit
#     - Verify the X force is rejected
#     - Verify the limited X force is applied
#     - Verify the limited X force is being measured
#     - Clear offset forces
#   - If the force actuator has an Y component
#     - Apply a Y force offset 15% higher than the max Y limit
#     - Verify the Y force is rejected
#     - Verify the limited Y force is applied
#     - Verify the limited Y force is being measured
#     - Apply a Y force offset 15% lower than the min Y limit
#     - Verify the Y force is rejected
#     - Verify the limited Y force is applied
#     - Verify the limited Y force is being measured
#     - Clear offset forces
#   - Apply a Z force offset 15% higher than the max Z limit
#   - Verify the Z force is rejected
#   - Verify the limited Z force is applied
#   - Verify the limited Z force is being measured
#   - Apply a Z force offset 15% lower than the min Z limit
#   - Verify the Z force is rejected
#   - Verify the limited Z force is applied
#   - Verify the limited Z force is being measured
#   - Clear offset forces
# - Transition from parked engineering state to standby
\end{verbatim}

\textbf{ Preconditions}:\\
\begin{enumerate}
\tightlist
\item
  All M1M3 Systems installed, except thermal control and interlocks
\item
  All M1M3 Systems operating, except thermal control and interlocks
\item
  The accelerometers on the mirror are operational and the outputs are
  being recorded.
\item
  The independent measuring system (IMS) is operational and the outputs
  are being recorded.
\item
  The M1M3 mirror system is in active mode and the mirror is resting on
  the actuators
\item
  The M1M3 mirror system is in passive mode and the mirror is resting on
  the static supports.
\item
  The M1M3 mirror positioning system is operational and the outputs are
  being recorded.
\item
  Hard points are in their nominal positions with negligible loads
  present.
\item
  M1M3 control software is current and configured for this test.
\item
  All test data is being recorded to the Engineering Facility Database
  EFD
\end{enumerate}

Execution status: {\bf Not Executed }

Final comment:\\


Detailed steps results:

\begin{longtable}{p{1cm}p{15cm}}
\hline
{Step} & Step Details\\ \hline
1 & Description \\
 & \begin{minipage}[t]{15cm}
{\footnotesize
Have software engineering update the ForceActuatorSettings.xml file to
specify ForceLimitXTablePath, ForceLimitYTablePath, and
ForceLimitZTablePath to utilize the ForceLimitXTableSmall.csv,
ForceLimitYTableSmall.csv, and ForceLimitZTableSmall.csv

\medskip }
\end{minipage}
\\ \cdashline{2-2}


 & Expected Result \\
 & \begin{minipage}[t]{15cm}{\footnotesize
N/A

\medskip }
\end{minipage} \\ \cdashline{2-2}

 & Actual Result \\
 & \begin{minipage}[t]{15cm}{\footnotesize

\medskip }
\end{minipage} \\ \cdashline{2-2}

 & Status: \textbf{ Not Executed } \\ \hline

2 & Description \\
 & \begin{minipage}[t]{15cm}
{\footnotesize
From a terminal run the following command after replacing yyyy with
year, mm with month, dd with day, hh with hour, and nn with
minute.\\[2\baselineskip]python36 M13T027.py \textbar{}\& tee
\textasciitilde{}/M13T027-yyyymmddhhnn.txt

\medskip }
\end{minipage}
\\ \cdashline{2-2}


 & Expected Result \\
 & \begin{minipage}[t]{15cm}{\footnotesize
N/A

\medskip }
\end{minipage} \\ \cdashline{2-2}

 & Actual Result \\
 & \begin{minipage}[t]{15cm}{\footnotesize

\medskip }
\end{minipage} \\ \cdashline{2-2}

 & Status: \textbf{ Not Executed } \\ \hline

3 & Description \\
 & \begin{minipage}[t]{15cm}
{\footnotesize
Wait for the script to complete.

\medskip }
\end{minipage}
\\ \cdashline{2-2}


 & Expected Result \\
 & \begin{minipage}[t]{15cm}{\footnotesize
N/A

\medskip }
\end{minipage} \\ \cdashline{2-2}

 & Actual Result \\
 & \begin{minipage}[t]{15cm}{\footnotesize

\medskip }
\end{minipage} \\ \cdashline{2-2}

 & Status: \textbf{ Not Executed } \\ \hline

4 & Description \\
 & \begin{minipage}[t]{15cm}
{\footnotesize
From a terminal run the following command replacing yyyy with year, mm
with month, dd with day, hh with hour, and nn with minute used in step
1.\\[2\baselineskip]grep -E ``ANOM -''
\textasciitilde{}/M13T027-yyyymmddhhnn.txt

\medskip }
\end{minipage}
\\ \cdashline{2-2}


 & Expected Result \\
 & \begin{minipage}[t]{15cm}{\footnotesize
The search should display no results meaning no anomalies occurred
during the test.

\medskip }
\end{minipage} \\ \cdashline{2-2}

 & Actual Result \\
 & \begin{minipage}[t]{15cm}{\footnotesize

\medskip }
\end{minipage} \\ \cdashline{2-2}

 & Status: \textbf{ Not Executed } \\ \hline

5 & Description \\
 & \begin{minipage}[t]{15cm}
{\footnotesize
Compress \textasciitilde{}/M13T027-yyyymmddhhnn.txt to
\textasciitilde{}/M13T027-yyyymmddhhnn.zip and attach it to this test
step.

\medskip }
\end{minipage}
\\ \cdashline{2-2}


 & Expected Result \\
 & \begin{minipage}[t]{15cm}{\footnotesize
N/A

\medskip }
\end{minipage} \\ \cdashline{2-2}

 & Actual Result \\
 & \begin{minipage}[t]{15cm}{\footnotesize

\medskip }
\end{minipage} \\ \cdashline{2-2}

 & Status: \textbf{ Not Executed } \\ \hline

6 & Description \\
 & \begin{minipage}[t]{15cm}
{\footnotesize
Have software engineering update the ForceActuatorSettings.xml file to
specify ForceLimitXTablePath, ForceLimitYTablePath, and
ForceLimitZTablePath to utilize the ForceLimitXTable.csv,
ForceLimitYTable.csv, and ForceLimitZTable.csv

\medskip }
\end{minipage}
\\ \cdashline{2-2}


 & Expected Result \\
 & \begin{minipage}[t]{15cm}{\footnotesize
N/A

\medskip }
\end{minipage} \\ \cdashline{2-2}

 & Actual Result \\
 & \begin{minipage}[t]{15cm}{\footnotesize

\medskip }
\end{minipage} \\ \cdashline{2-2}

 & Status: \textbf{ Not Executed } \\ \hline

\end{longtable}

\paragraph{Test Case LVV-T242 - M13T-028: Actuator to Actuator Force Delta for 6 nearest neighbors }\mbox{}\\

Open  \href{https://jira.lsstcorp.org/secure/Tests.jspa#/testCase/LVV-T242}{\textit{ LVV-T242 } }
test case in Jira.

The safe force delta is the safe limit of variation in force between
adjacent actuators between the 6 nearest neighbors. This limit was
determined thru analysis. The purpose of this test is to verify that the
system will trigger a fault if the force delta is greater than 90\% of
the safe value, and a fault if the Safe Force value is exceeded.

\textbf{ Preconditions}:\\
\begin{enumerate}
\tightlist
\item
  All M1M3 Systems installed, except thermal control and interlocks
\item
  All M1M3 Systems operating, except thermal control and interlocks.
\item
  The M1M3 mirror system is in active mode and the mirror is resting on
  the actuators
\item
  Hard points are in their nominal positions with negligible loads
  present.
\item
  M1M3 control software is current and configured for this test.
\item
  All test data is being recorded to the Engineering Facility Database
  EFD
\end{enumerate}

Execution status: {\bf Not Executed }

Final comment:\\


Detailed steps results:

\begin{longtable}{p{1cm}p{15cm}}
\hline
{Step} & Step Details\\ \hline
1 & Description \\
 & \begin{minipage}[t]{15cm}
{\footnotesize
Have software engineering update the SafetyControllerSettings.xml to
make it so FaultOnNearNeighborCheck is 0.

\medskip }
\end{minipage}
\\ \cdashline{2-2}


 & Expected Result \\
 & \begin{minipage}[t]{15cm}{\footnotesize
N/A

\medskip }
\end{minipage} \\ \cdashline{2-2}

 & Actual Result \\
 & \begin{minipage}[t]{15cm}{\footnotesize

\medskip }
\end{minipage} \\ \cdashline{2-2}

 & Status: \textbf{ Not Executed } \\ \hline

2 & Description \\
 & \begin{minipage}[t]{15cm}
{\footnotesize
From a terminal run the following command after replacing yyyy with
year, mm with month, dd with day, hh with hour, and nn with
minute.\\[2\baselineskip]python36 M13T028.py \textbar{}\& tee
\textasciitilde{}/M13T028-yyyymmddhhnn.txt

\medskip }
\end{minipage}
\\ \cdashline{2-2}


 & Expected Result \\
 & \begin{minipage}[t]{15cm}{\footnotesize
N/A

\medskip }
\end{minipage} \\ \cdashline{2-2}

 & Actual Result \\
 & \begin{minipage}[t]{15cm}{\footnotesize

\medskip }
\end{minipage} \\ \cdashline{2-2}

 & Status: \textbf{ Not Executed } \\ \hline

3 & Description \\
 & \begin{minipage}[t]{15cm}
{\footnotesize
Wait for the script to complete.

\medskip }
\end{minipage}
\\ \cdashline{2-2}


 & Expected Result \\
 & \begin{minipage}[t]{15cm}{\footnotesize
N/A

\medskip }
\end{minipage} \\ \cdashline{2-2}

 & Actual Result \\
 & \begin{minipage}[t]{15cm}{\footnotesize

\medskip }
\end{minipage} \\ \cdashline{2-2}

 & Status: \textbf{ Not Executed } \\ \hline

4 & Description \\
 & \begin{minipage}[t]{15cm}
{\footnotesize
From a terminal run the following command replacing yyyy with year, mm
with month, dd with day, hh with hour, and nn with minute used in step
1.\\[2\baselineskip]grep -E ``ANOM -''
\textasciitilde{}/M13T028-yyyymmddhhnn.txt

\medskip }
\end{minipage}
\\ \cdashline{2-2}


 & Expected Result \\
 & \begin{minipage}[t]{15cm}{\footnotesize
The search should display no results meaning no anomalies occurred
during the test.

\medskip }
\end{minipage} \\ \cdashline{2-2}

 & Actual Result \\
 & \begin{minipage}[t]{15cm}{\footnotesize

\medskip }
\end{minipage} \\ \cdashline{2-2}

 & Status: \textbf{ Not Executed } \\ \hline

5 & Description \\
 & \begin{minipage}[t]{15cm}
{\footnotesize
Compress \textasciitilde{}/M13T028-yyyymmddhhnn.txt to
\textasciitilde{}/M13T028-yyyymmddhhnn.zip and attach it to this test
step.

\medskip }
\end{minipage}
\\ \cdashline{2-2}


 & Expected Result \\
 & \begin{minipage}[t]{15cm}{\footnotesize
N/A

\medskip }
\end{minipage} \\ \cdashline{2-2}

 & Actual Result \\
 & \begin{minipage}[t]{15cm}{\footnotesize

\medskip }
\end{minipage} \\ \cdashline{2-2}

 & Status: \textbf{ Not Executed } \\ \hline

6 & Description \\
 & \begin{minipage}[t]{15cm}
{\footnotesize
Have software engineering update the SafetyControllerSettings.xml to
make it so FaultOnNearNeighborCheck is returned to 1.

\medskip }
\end{minipage}
\\ \cdashline{2-2}


 & Expected Result \\
 & \begin{minipage}[t]{15cm}{\footnotesize
N/A

\medskip }
\end{minipage} \\ \cdashline{2-2}

 & Actual Result \\
 & \begin{minipage}[t]{15cm}{\footnotesize

\medskip }
\end{minipage} \\ \cdashline{2-2}

 & Status: \textbf{ Not Executed } \\ \hline

\end{longtable}

\paragraph{Test Case LVV-T237 - M13T-017: Active Optics Actuator Force Updates }\mbox{}\\

Open  \href{https://jira.lsstcorp.org/secure/Tests.jspa#/testCase/LVV-T237}{\textit{ LVV-T237 } }
test case in Jira.

The purpose of this test is verify that the active optical forces change
only when specifically requested to. When new AOS forces are sent, they
replace the ones previous.

\textbf{ Preconditions}:\\
This is a software simulation that should be performed before actual
testing

Execution status: {\bf Not Executed }

Final comment:\\


Detailed steps results:

\begin{longtable}{p{1cm}p{15cm}}
\hline
{Step} & Step Details\\ \hline
1 & Description \\
 & \begin{minipage}[t]{15cm}
{\footnotesize
From a terminal run the following command after replacing yyyy with
year, mm with month, dd with day, hh with hour, and nn with
minute.\\[2\baselineskip]python36 M13T017.py \textbar{}\& tee
\textasciitilde{}/M13T017-yyyymmddhhnn.txt

\medskip }
\end{minipage}
\\ \cdashline{2-2}


 & Expected Result \\
 & \begin{minipage}[t]{15cm}{\footnotesize
N/A

\medskip }
\end{minipage} \\ \cdashline{2-2}

 & Actual Result \\
 & \begin{minipage}[t]{15cm}{\footnotesize

\medskip }
\end{minipage} \\ \cdashline{2-2}

 & Status: \textbf{ Not Executed } \\ \hline

2 & Description \\
 & \begin{minipage}[t]{15cm}
{\footnotesize
Wait for the script to complete.

\medskip }
\end{minipage}
\\ \cdashline{2-2}


 & Expected Result \\
 & \begin{minipage}[t]{15cm}{\footnotesize
N/A

\medskip }
\end{minipage} \\ \cdashline{2-2}

 & Actual Result \\
 & \begin{minipage}[t]{15cm}{\footnotesize

\medskip }
\end{minipage} \\ \cdashline{2-2}

 & Status: \textbf{ Not Executed } \\ \hline

3 & Description \\
 & \begin{minipage}[t]{15cm}
{\footnotesize
From a terminal run the following command replacing yyyy with year, mm
with month, dd with day, hh with hour, and nn with minute used in step
1.\\[2\baselineskip]grep -E ``ANOM -''
\textasciitilde{}/M13T017-yyyymmddhhnn.txt

\medskip }
\end{minipage}
\\ \cdashline{2-2}


 & Expected Result \\
 & \begin{minipage}[t]{15cm}{\footnotesize
The search should display no results meaning no anomalies occurred
during the test.

\medskip }
\end{minipage} \\ \cdashline{2-2}

 & Actual Result \\
 & \begin{minipage}[t]{15cm}{\footnotesize

\medskip }
\end{minipage} \\ \cdashline{2-2}

 & Status: \textbf{ Not Executed } \\ \hline

4 & Description \\
 & \begin{minipage}[t]{15cm}
{\footnotesize
Compress \textasciitilde{}/M13T017-yyyymmddhhnn.txt to
\textasciitilde{}/M13T017-yyyymmddhhnn.zip and attach it to this test
step.

\medskip }
\end{minipage}
\\ \cdashline{2-2}


 & Expected Result \\
 & \begin{minipage}[t]{15cm}{\footnotesize
N/A

\medskip }
\end{minipage} \\ \cdashline{2-2}

 & Actual Result \\
 & \begin{minipage}[t]{15cm}{\footnotesize

\medskip }
\end{minipage} \\ \cdashline{2-2}

 & Status: \textbf{ Not Executed } \\ \hline

\end{longtable}

\paragraph{Test Case LVV-T236 - M13T-016: Static Force Correction }\mbox{}\\

Open  \href{https://jira.lsstcorp.org/secure/Tests.jspa#/testCase/LVV-T236}{\textit{ LVV-T236 } }
test case in Jira.

A portion of the actuator force budget is allocated to static
correction. The static correction is principally used to correct the
variation between the theoretically perfect mirror and the as-built
mirror. The initial force values are provided through mirror lab
testing. They will be periodically refined by measurements.\\
\hspace*{0.333em}\\
When the M1M3 is in its operational configuration, the actuators forces
are principally determined by a LUT (look up table). Offsets are added
to the LUT to produce the combined force command. The static correction
is one of these offsets. Since the purpose of the static correction is
to deform the mirror, the static offset correction forces should produce
no net forces or moments. During this test, the force balance system
will be operational so the net forces and moments can be verified

\textbf{ Preconditions}:\\
\begin{enumerate}
\tightlist
\item
  All M1M3 Systems installed, except thermal control and interlocks
\item
  All M1M3 Systems operating, except thermal control and interlocks
\item
  The accelerometers on the mirror are operational and the outputs are
  being recorded.
\item
  The independent measuring system (IMS) is operational and the outputs
  are being recorded.
\item
  The M1M3 mirror system is in active mode and the mirror is resting on
  the actuators
\item
  The M1M3 mirror positioning system is operational and the outputs are
  being recorded.
\item
  Hard points are in their nominal positions with negligible loads
  present.
\item
  M1M3 control software is current and configured for this test.
\item
  All test data is being recorded to the Engineering Facility Database
  EFD
\end{enumerate}

Execution status: {\bf Not Executed }

Final comment:\\


Detailed steps results:

\begin{longtable}{p{1cm}p{15cm}}
\hline
{Step} & Step Details\\ \hline
1 & Description \\
 & \begin{minipage}[t]{15cm}
{\footnotesize
Have software engineering update the ForceActuatorSettings.xml file to
specify StaticXTablePath, StaticYTablePath, and StaticZTablePath to
utilize the StaticXTableNone.csv, StaticYTableNone.csv, and
StaticZTableNone.csv

\medskip }
\end{minipage}
\\ \cdashline{2-2}


 & Expected Result \\
 & \begin{minipage}[t]{15cm}{\footnotesize

\medskip }
\end{minipage} \\ \cdashline{2-2}

 & Actual Result \\
 & \begin{minipage}[t]{15cm}{\footnotesize

\medskip }
\end{minipage} \\ \cdashline{2-2}

 & Status: \textbf{ Not Executed } \\ \hline

2 & Description \\
 & \begin{minipage}[t]{15cm}
{\footnotesize
In the EUI click the Start button.

\medskip }
\end{minipage}
\\ \cdashline{2-2}


 & Expected Result \\
 & \begin{minipage}[t]{15cm}{\footnotesize

\medskip }
\end{minipage} \\ \cdashline{2-2}

 & Actual Result \\
 & \begin{minipage}[t]{15cm}{\footnotesize

\medskip }
\end{minipage} \\ \cdashline{2-2}

 & Status: \textbf{ Not Executed } \\ \hline

3 & Description \\
 & \begin{minipage}[t]{15cm}
{\footnotesize
In the EUI click the Enable button.

\medskip }
\end{minipage}
\\ \cdashline{2-2}


 & Expected Result \\
 & \begin{minipage}[t]{15cm}{\footnotesize

\medskip }
\end{minipage} \\ \cdashline{2-2}

 & Actual Result \\
 & \begin{minipage}[t]{15cm}{\footnotesize

\medskip }
\end{minipage} \\ \cdashline{2-2}

 & Status: \textbf{ Not Executed } \\ \hline

4 & Description \\
 & \begin{minipage}[t]{15cm}
{\footnotesize
In the EUI click the Enter Engineering button.

\medskip }
\end{minipage}
\\ \cdashline{2-2}


 & Expected Result \\
 & \begin{minipage}[t]{15cm}{\footnotesize

\medskip }
\end{minipage} \\ \cdashline{2-2}

 & Actual Result \\
 & \begin{minipage}[t]{15cm}{\footnotesize

\medskip }
\end{minipage} \\ \cdashline{2-2}

 & Status: \textbf{ Not Executed } \\ \hline

5 & Description \\
 & \begin{minipage}[t]{15cm}
{\footnotesize
In the EUI click the Raise Mirror button.

\medskip }
\end{minipage}
\\ \cdashline{2-2}


 & Expected Result \\
 & \begin{minipage}[t]{15cm}{\footnotesize

\medskip }
\end{minipage} \\ \cdashline{2-2}

 & Actual Result \\
 & \begin{minipage}[t]{15cm}{\footnotesize

\medskip }
\end{minipage} \\ \cdashline{2-2}

 & Status: \textbf{ Not Executed } \\ \hline

6 & Description \\
 & \begin{minipage}[t]{15cm}
{\footnotesize
Wait until the detailed state reports ActiveState.

\medskip }
\end{minipage}
\\ \cdashline{2-2}


 & Expected Result \\
 & \begin{minipage}[t]{15cm}{\footnotesize

\medskip }
\end{minipage} \\ \cdashline{2-2}

 & Actual Result \\
 & \begin{minipage}[t]{15cm}{\footnotesize

\medskip }
\end{minipage} \\ \cdashline{2-2}

 & Status: \textbf{ Not Executed } \\ \hline

7 & Description \\
 & \begin{minipage}[t]{15cm}
{\footnotesize
In the EUI under the Hardpoints tab click the Disable Hardpoint
Corrections Process button.

\medskip }
\end{minipage}
\\ \cdashline{2-2}


 & Expected Result \\
 & \begin{minipage}[t]{15cm}{\footnotesize

\medskip }
\end{minipage} \\ \cdashline{2-2}

 & Actual Result \\
 & \begin{minipage}[t]{15cm}{\footnotesize

\medskip }
\end{minipage} \\ \cdashline{2-2}

 & Status: \textbf{ Not Executed } \\ \hline

8 & Description \\
 & \begin{minipage}[t]{15cm}
{\footnotesize
Take a screen shot of the AppliedStaticForces and of the HardpointForces

\medskip }
\end{minipage}
\\ \cdashline{2-2}


 & Expected Result \\
 & \begin{minipage}[t]{15cm}{\footnotesize

\medskip }
\end{minipage} \\ \cdashline{2-2}

 & Actual Result \\
 & \begin{minipage}[t]{15cm}{\footnotesize

\medskip }
\end{minipage} \\ \cdashline{2-2}

 & Status: \textbf{ Not Executed } \\ \hline

9 & Description \\
 & \begin{minipage}[t]{15cm}
{\footnotesize
In the EUI click the Lower Mirror button.

\medskip }
\end{minipage}
\\ \cdashline{2-2}


 & Expected Result \\
 & \begin{minipage}[t]{15cm}{\footnotesize

\medskip }
\end{minipage} \\ \cdashline{2-2}

 & Actual Result \\
 & \begin{minipage}[t]{15cm}{\footnotesize

\medskip }
\end{minipage} \\ \cdashline{2-2}

 & Status: \textbf{ Not Executed } \\ \hline

10 & Description \\
 & \begin{minipage}[t]{15cm}
{\footnotesize
Wait until the detailed state reports ParkedState.

\medskip }
\end{minipage}
\\ \cdashline{2-2}


 & Expected Result \\
 & \begin{minipage}[t]{15cm}{\footnotesize

\medskip }
\end{minipage} \\ \cdashline{2-2}

 & Actual Result \\
 & \begin{minipage}[t]{15cm}{\footnotesize

\medskip }
\end{minipage} \\ \cdashline{2-2}

 & Status: \textbf{ Not Executed } \\ \hline

11 & Description \\
 & \begin{minipage}[t]{15cm}
{\footnotesize
In the EUI click the Disable button.

\medskip }
\end{minipage}
\\ \cdashline{2-2}


 & Expected Result \\
 & \begin{minipage}[t]{15cm}{\footnotesize

\medskip }
\end{minipage} \\ \cdashline{2-2}

 & Actual Result \\
 & \begin{minipage}[t]{15cm}{\footnotesize

\medskip }
\end{minipage} \\ \cdashline{2-2}

 & Status: \textbf{ Not Executed } \\ \hline

12 & Description \\
 & \begin{minipage}[t]{15cm}
{\footnotesize
In the EUI click the Standby button.

\medskip }
\end{minipage}
\\ \cdashline{2-2}


 & Expected Result \\
 & \begin{minipage}[t]{15cm}{\footnotesize

\medskip }
\end{minipage} \\ \cdashline{2-2}

 & Actual Result \\
 & \begin{minipage}[t]{15cm}{\footnotesize

\medskip }
\end{minipage} \\ \cdashline{2-2}

 & Status: \textbf{ Not Executed } \\ \hline

13 & Description \\
 & \begin{minipage}[t]{15cm}
{\footnotesize
Have software engineering update the ForceActuatorSettings.xml file to
specify StaticXTablePath, StaticYTablePath, and StaticZTablePath to
utilize the StaticXTable.csv, StaticYTable.csv, and StaticZTable.csv

\medskip }
\end{minipage}
\\ \cdashline{2-2}


 & Expected Result \\
 & \begin{minipage}[t]{15cm}{\footnotesize

\medskip }
\end{minipage} \\ \cdashline{2-2}

 & Actual Result \\
 & \begin{minipage}[t]{15cm}{\footnotesize

\medskip }
\end{minipage} \\ \cdashline{2-2}

 & Status: \textbf{ Not Executed } \\ \hline

14 & Description \\
 & \begin{minipage}[t]{15cm}
{\footnotesize
In the EUI click the Start button.

\medskip }
\end{minipage}
\\ \cdashline{2-2}


 & Expected Result \\
 & \begin{minipage}[t]{15cm}{\footnotesize

\medskip }
\end{minipage} \\ \cdashline{2-2}

 & Actual Result \\
 & \begin{minipage}[t]{15cm}{\footnotesize

\medskip }
\end{minipage} \\ \cdashline{2-2}

 & Status: \textbf{ Not Executed } \\ \hline

15 & Description \\
 & \begin{minipage}[t]{15cm}
{\footnotesize
In the EUI click the Enable button.

\medskip }
\end{minipage}
\\ \cdashline{2-2}


 & Expected Result \\
 & \begin{minipage}[t]{15cm}{\footnotesize

\medskip }
\end{minipage} \\ \cdashline{2-2}

 & Actual Result \\
 & \begin{minipage}[t]{15cm}{\footnotesize

\medskip }
\end{minipage} \\ \cdashline{2-2}

 & Status: \textbf{ Not Executed } \\ \hline

16 & Description \\
 & \begin{minipage}[t]{15cm}
{\footnotesize
In the EUI click the Enter Engineering button.

\medskip }
\end{minipage}
\\ \cdashline{2-2}


 & Expected Result \\
 & \begin{minipage}[t]{15cm}{\footnotesize

\medskip }
\end{minipage} \\ \cdashline{2-2}

 & Actual Result \\
 & \begin{minipage}[t]{15cm}{\footnotesize

\medskip }
\end{minipage} \\ \cdashline{2-2}

 & Status: \textbf{ Not Executed } \\ \hline

17 & Description \\
 & \begin{minipage}[t]{15cm}
{\footnotesize
In the EUI click the Raise Mirror button.

\medskip }
\end{minipage}
\\ \cdashline{2-2}


 & Expected Result \\
 & \begin{minipage}[t]{15cm}{\footnotesize

\medskip }
\end{minipage} \\ \cdashline{2-2}

 & Actual Result \\
 & \begin{minipage}[t]{15cm}{\footnotesize

\medskip }
\end{minipage} \\ \cdashline{2-2}

 & Status: \textbf{ Not Executed } \\ \hline

18 & Description \\
 & \begin{minipage}[t]{15cm}
{\footnotesize
Wait until the detailed state reports ActiveState.

\medskip }
\end{minipage}
\\ \cdashline{2-2}


 & Expected Result \\
 & \begin{minipage}[t]{15cm}{\footnotesize

\medskip }
\end{minipage} \\ \cdashline{2-2}

 & Actual Result \\
 & \begin{minipage}[t]{15cm}{\footnotesize

\medskip }
\end{minipage} \\ \cdashline{2-2}

 & Status: \textbf{ Not Executed } \\ \hline

19 & Description \\
 & \begin{minipage}[t]{15cm}
{\footnotesize
In the EUI under the Hardpoints tab click the Disable Hardpoint
Corrections Process button.

\medskip }
\end{minipage}
\\ \cdashline{2-2}


 & Expected Result \\
 & \begin{minipage}[t]{15cm}{\footnotesize

\medskip }
\end{minipage} \\ \cdashline{2-2}

 & Actual Result \\
 & \begin{minipage}[t]{15cm}{\footnotesize

\medskip }
\end{minipage} \\ \cdashline{2-2}

 & Status: \textbf{ Not Executed } \\ \hline

20 & Description \\
 & \begin{minipage}[t]{15cm}
{\footnotesize
Take a screen shot of the AppliedStaticForces and of the HardpointForces

\medskip }
\end{minipage}
\\ \cdashline{2-2}


 & Expected Result \\
 & \begin{minipage}[t]{15cm}{\footnotesize

\medskip }
\end{minipage} \\ \cdashline{2-2}

 & Actual Result \\
 & \begin{minipage}[t]{15cm}{\footnotesize

\medskip }
\end{minipage} \\ \cdashline{2-2}

 & Status: \textbf{ Not Executed } \\ \hline

21 & Description \\
 & \begin{minipage}[t]{15cm}
{\footnotesize
In the EUI click the Lower Mirror button.

\medskip }
\end{minipage}
\\ \cdashline{2-2}


 & Expected Result \\
 & \begin{minipage}[t]{15cm}{\footnotesize

\medskip }
\end{minipage} \\ \cdashline{2-2}

 & Actual Result \\
 & \begin{minipage}[t]{15cm}{\footnotesize

\medskip }
\end{minipage} \\ \cdashline{2-2}

 & Status: \textbf{ Not Executed } \\ \hline

22 & Description \\
 & \begin{minipage}[t]{15cm}
{\footnotesize
Wait until the detailed state reports ParkedState.

\medskip }
\end{minipage}
\\ \cdashline{2-2}


 & Expected Result \\
 & \begin{minipage}[t]{15cm}{\footnotesize

\medskip }
\end{minipage} \\ \cdashline{2-2}

 & Actual Result \\
 & \begin{minipage}[t]{15cm}{\footnotesize

\medskip }
\end{minipage} \\ \cdashline{2-2}

 & Status: \textbf{ Not Executed } \\ \hline

23 & Description \\
 & \begin{minipage}[t]{15cm}
{\footnotesize
In the EUI click the Disable button.

\medskip }
\end{minipage}
\\ \cdashline{2-2}


 & Expected Result \\
 & \begin{minipage}[t]{15cm}{\footnotesize

\medskip }
\end{minipage} \\ \cdashline{2-2}

 & Actual Result \\
 & \begin{minipage}[t]{15cm}{\footnotesize

\medskip }
\end{minipage} \\ \cdashline{2-2}

 & Status: \textbf{ Not Executed } \\ \hline

24 & Description \\
 & \begin{minipage}[t]{15cm}
{\footnotesize
In the EUI click the Standby button.

\medskip }
\end{minipage}
\\ \cdashline{2-2}


 & Expected Result \\
 & \begin{minipage}[t]{15cm}{\footnotesize

\medskip }
\end{minipage} \\ \cdashline{2-2}

 & Actual Result \\
 & \begin{minipage}[t]{15cm}{\footnotesize

\medskip }
\end{minipage} \\ \cdashline{2-2}

 & Status: \textbf{ Not Executed } \\ \hline

25 & Description \\
 & \begin{minipage}[t]{15cm}
{\footnotesize
Verify the applied static forces changes between the two configurations
and the hardpoint forces are closer to zero with the application of the
static forces.

\medskip }
\end{minipage}
\\ \cdashline{2-2}


 & Expected Result \\
 & \begin{minipage}[t]{15cm}{\footnotesize
Applied static forces are different.\\
Hardpoint forces are closer to 0N than before.

\medskip }
\end{minipage} \\ \cdashline{2-2}

 & Actual Result \\
 & \begin{minipage}[t]{15cm}{\footnotesize

\medskip }
\end{minipage} \\ \cdashline{2-2}

 & Status: \textbf{ Not Executed } \\ \hline

\end{longtable}

\paragraph{Test Case LVV-T257 - Position Stability During Active Mode Operation }\mbox{}\\

Open  \href{https://jira.lsstcorp.org/secure/Tests.jspa#/testCase/LVV-T257}{\textit{ LVV-T257 } }
test case in Jira.

Motions of the M1M3 at frequencies greater than \textasciitilde{}1Hz
cannot be compensated by the telescope and produce image degradation.
Small low speed motions of less than \textasciitilde{}1 Hz can be
compensated for by the telescope pointing system operating the TMA. It
is the displacements and not the accelerations that actually produce the
image degradation. Consequently, low frequency vibrations are of greater
interest since the displacement varies with the inverse square of the
frequency.\\
\hspace*{0.333em}\\
Most of the motion is expected to be produce be either slow speed
vibrations of less than the fundamental frequency of \textasciitilde{}12
Hz, or motions near the fundamental frequency. Consequently, ~the
frequencies most likely to produce image degradation are those between
\textasciitilde{}1 Hz and \textasciitilde{}24 Hz. ~\\
\hspace*{0.333em}\\
Since the displacements are large relative to the accelerations, it is
difficult to determine low frequency displacements with an
accelerometer. Since they are mechanical elements it is difficult to
determine high frequency displacement with the sensor of the IMS.
Consequently both systems need to be employed to verify this
requirements. The IMS will be used to verify motions between
\textasciitilde{}1 and 5 Hz. The accelerometers will be used to verify
motions between 5 and 24 Hz.

\textbf{ Preconditions}:\\
\begin{longtable}[]{@{}l@{}}
\toprule
\begin{minipage}[t]{0.97\columnwidth}\raggedright\strut
Note: This test can be significantly corrupted by back ground
vibrations. Consequently, it is recommended that this test be performed
late at night when there is a minimum of activity. Any unnecessary
equipment should be turned off, etc.\\

\begin{enumerate}
\tightlist
\item
  All M1M3 Systems installed, except thermal control and interlocks
\item
  All M1M3 Systems operating, except thermal control and interlocks
\item
  The accelerometers on the mirror are operational and the outputs are
  being recorded.
\item
  The independent measuring system (IMS) is operational and the outputs
  are being recorded.
\item
  The M1M3 mirror system is in active mode and the mirror is resting on
  the actuators
\item
  The M1M3 mirror positioning system is operational and the outputs are
  being recorded.
\item
  Hard points are in their nominal positions with negligible loads
  present.
\item
  M1M3 control software is current and configured for this test.
\item
  All test data is being recorded to the Engineering Facility Database
  EFD
\end{enumerate}\strut
\end{minipage}\tabularnewline
\bottomrule
\end{longtable}

Execution status: {\bf Not Executed }

Final comment:\\


Detailed steps results:

\begin{longtable}{p{1cm}p{15cm}}
\hline
{Step} & Step Details\\ \hline
1 & Description \\
 & \begin{minipage}[t]{15cm}
{\footnotesize
\begin{enumerate}
\tightlist
\item
  Put in active mode, Raise mirror .001m (1mm, but units of meters
  needed to input to the software)
\item
  Wait 5 seconds
\item
  Start collecting data for a random 15 second interval, monitor the IMS
  and mirror accelerometers
\item
  Repeat 3x in the following locations:

  \begin{enumerate}
  \tightlist
  \item
    0, 0, 0
  \item
    +X, 0, 0
  \item
    -X, 0, 0
  \item
    0,+Y, 0
  \item
    0, -Y, 0
  \item
    0, 0, +Z
  \item
    0, 0, -Z
  \item
    +X, 0, +Z
  \item
    +X, 0, -Z
  \item
    - X, 0, +Z
  \item
    - X, 0, -Z
  \item
    0, +Y, +Z
  \item
    0, +Y, -Z
  \item
    0, -Y, +Z
  \item
    0, -Y, -Z~
  \item
    +X, +Y, 0
  \item
    -X, +Y, 0
  \item
    +X, -Y, 0
  \item
    -X, -Y, 0
  \end{enumerate}
\item
  To retrieve the data, look at the position command for each of the
  steps and then use a sample of 15 seconds from that data.
\item
  Calculate the RMS

  \begin{enumerate}
  \tightlist
  \item
    average 3 values, variation from average and those values is
    repeatability, calculate RMS from that.
  \end{enumerate}
\end{enumerate}

\medskip }
\end{minipage}
\\ \cdashline{2-2}


 & Expected Result \\
 & \begin{minipage}[t]{15cm}{\footnotesize

\medskip }
\end{minipage} \\ \cdashline{2-2}

 & Actual Result \\
 & \begin{minipage}[t]{15cm}{\footnotesize

\medskip }
\end{minipage} \\ \cdashline{2-2}

 & Status: \textbf{ Not Executed } \\ \hline

2 & Description \\
 & \begin{minipage}[t]{15cm}
{\footnotesize

\medskip }
\end{minipage}
\\ \cdashline{2-2}


 & Expected Result \\
 & \begin{minipage}[t]{15cm}{\footnotesize
The total motion determined through the combined measurements of the IMS
and the accelerometers should be less than the requirement.\\
Since there is overlap in the motions determined by the IMS and the
motions determined by the mirror accelerometers, accurate determination
of this motion will be computationally complex and will require
significant post processing.\\
However, the motions determined by the individual systems are fairly
straightforward.\\
Addition of these motions by RSS is considered
adequate.\\[2\baselineskip]To be verified by subsequent analysis of the
data collected and stored in the EFD.\\[2\baselineskip]\textbf{Needs to
be clarified}\\
\textbf{RMS compared to what tolerance.}

\medskip }
\end{minipage} \\ \cdashline{2-2}

 & Actual Result \\
 & \begin{minipage}[t]{15cm}{\footnotesize

\medskip }
\end{minipage} \\ \cdashline{2-2}

 & Status: \textbf{ Not Executed } \\ \hline

\end{longtable}

\paragraph{Test Case LVV-T245 - M13T-031: Automatic Shutdown }\mbox{}\\

Open  \href{https://jira.lsstcorp.org/secure/Tests.jspa#/testCase/LVV-T245}{\textit{ LVV-T245 } }
test case in Jira.

TBD

\textbf{ Preconditions}:\\
\begin{enumerate}
\tightlist
\item
  All M1M3 Systems installed, except thermal control and interlocks
\item
  All M1M3 Systems operating, except thermal control and interlocks
\item
  The accelerometers on the mirror are operational and the outputs are
  being recorded.
\item
  The independent measuring system (IMS) is operational and the outputs
  are being recorded.
\item
  The M1M3 mirror system is in active mode and the mirror is resting on
  the actuators
\item
  The M1M3 mirror system is in passive mode and the mirror is resting on
  the static supports.
\item
  The M1M3 mirror positioning system is operational and the outputs are
  being recorded.
\item
  Hard points are in their nominal positions with negligible loads
  present.
\item
  M1M3 control software is current and configured for this test.
\item
  All test data is being recorded to the Engineering Facility Database
  EFD
\end{enumerate}

Execution status: {\bf Not Executed }

Final comment:\\


Detailed steps results:

\begin{longtable}{p{1cm}p{15cm}}
\hline
{Step} & Step Details\\ \hline
1 & Description \\
 & \begin{minipage}[t]{15cm}
{\footnotesize
In the EUI click the Start button.

\medskip }
\end{minipage}
\\ \cdashline{2-2}


 & Expected Result \\
 & \begin{minipage}[t]{15cm}{\footnotesize
N/A

\medskip }
\end{minipage} \\ \cdashline{2-2}

 & Actual Result \\
 & \begin{minipage}[t]{15cm}{\footnotesize

\medskip }
\end{minipage} \\ \cdashline{2-2}

 & Status: \textbf{ Not Executed } \\ \hline

2 & Description \\
 & \begin{minipage}[t]{15cm}
{\footnotesize
In the EUI click the Enable button.

\medskip }
\end{minipage}
\\ \cdashline{2-2}


 & Expected Result \\
 & \begin{minipage}[t]{15cm}{\footnotesize
N/A

\medskip }
\end{minipage} \\ \cdashline{2-2}

 & Actual Result \\
 & \begin{minipage}[t]{15cm}{\footnotesize

\medskip }
\end{minipage} \\ \cdashline{2-2}

 & Status: \textbf{ Not Executed } \\ \hline

3 & Description \\
 & \begin{minipage}[t]{15cm}
{\footnotesize
In the EUI click the Raise Mirror button.

\medskip }
\end{minipage}
\\ \cdashline{2-2}


 & Expected Result \\
 & \begin{minipage}[t]{15cm}{\footnotesize
N/A

\medskip }
\end{minipage} \\ \cdashline{2-2}

 & Actual Result \\
 & \begin{minipage}[t]{15cm}{\footnotesize

\medskip }
\end{minipage} \\ \cdashline{2-2}

 & Status: \textbf{ Not Executed } \\ \hline

4 & Description \\
 & \begin{minipage}[t]{15cm}
{\footnotesize
Watch the IMS screen and wait for the Z position to change by at least
2mm.

\medskip }
\end{minipage}
\\ \cdashline{2-2}


 & Expected Result \\
 & \begin{minipage}[t]{15cm}{\footnotesize
N/A

\medskip }
\end{minipage} \\ \cdashline{2-2}

 & Actual Result \\
 & \begin{minipage}[t]{15cm}{\footnotesize

\medskip }
\end{minipage} \\ \cdashline{2-2}

 & Status: \textbf{ Not Executed } \\ \hline

5 & Description \\
 & \begin{minipage}[t]{15cm}
{\footnotesize
Start collecting IMS data.

\medskip }
\end{minipage}
\\ \cdashline{2-2}


 & Expected Result \\
 & \begin{minipage}[t]{15cm}{\footnotesize
N/A

\medskip }
\end{minipage} \\ \cdashline{2-2}

 & Actual Result \\
 & \begin{minipage}[t]{15cm}{\footnotesize

\medskip }
\end{minipage} \\ \cdashline{2-2}

 & Status: \textbf{ Not Executed } \\ \hline

6 & Description \\
 & \begin{minipage}[t]{15cm}
{\footnotesize
In the controls cabinet flip the switch off for power subnet A.

\medskip }
\end{minipage}
\\ \cdashline{2-2}


 & Expected Result \\
 & \begin{minipage}[t]{15cm}{\footnotesize
N/A

\medskip }
\end{minipage} \\ \cdashline{2-2}

 & Actual Result \\
 & \begin{minipage}[t]{15cm}{\footnotesize

\medskip }
\end{minipage} \\ \cdashline{2-2}

 & Status: \textbf{ Not Executed } \\ \hline

7 & Description \\
 & \begin{minipage}[t]{15cm}
{\footnotesize
In the EUI verify an error occurs and an error code of 6110 / 1 is
observed. This error code indicates that ILC communication has been lost
to at least 1 ILC for over 60ms.

\medskip }
\end{minipage}
\\ \cdashline{2-2}


 & Expected Result \\
 & \begin{minipage}[t]{15cm}{\footnotesize
Error Code 6110 with detailed code of 1 is observed.

\medskip }
\end{minipage} \\ \cdashline{2-2}

 & Actual Result \\
 & \begin{minipage}[t]{15cm}{\footnotesize

\medskip }
\end{minipage} \\ \cdashline{2-2}

 & Status: \textbf{ Not Executed } \\ \hline

8 & Description \\
 & \begin{minipage}[t]{15cm}
{\footnotesize
Stop collecting IMS data.

\medskip }
\end{minipage}
\\ \cdashline{2-2}


 & Expected Result \\
 & \begin{minipage}[t]{15cm}{\footnotesize
N/A

\medskip }
\end{minipage} \\ \cdashline{2-2}

 & Actual Result \\
 & \begin{minipage}[t]{15cm}{\footnotesize

\medskip }
\end{minipage} \\ \cdashline{2-2}

 & Status: \textbf{ Not Executed } \\ \hline

9 & Description \\
 & \begin{minipage}[t]{15cm}
{\footnotesize
In the controls cabinet flip the switch on for power subnet A.

\medskip }
\end{minipage}
\\ \cdashline{2-2}


 & Expected Result \\
 & \begin{minipage}[t]{15cm}{\footnotesize
N/A

\medskip }
\end{minipage} \\ \cdashline{2-2}

 & Actual Result \\
 & \begin{minipage}[t]{15cm}{\footnotesize

\medskip }
\end{minipage} \\ \cdashline{2-2}

 & Status: \textbf{ Not Executed } \\ \hline

10 & Description \\
 & \begin{minipage}[t]{15cm}
{\footnotesize
In the EUI click the Standby button.

\medskip }
\end{minipage}
\\ \cdashline{2-2}


 & Expected Result \\
 & \begin{minipage}[t]{15cm}{\footnotesize
N/A

\medskip }
\end{minipage} \\ \cdashline{2-2}

 & Actual Result \\
 & \begin{minipage}[t]{15cm}{\footnotesize

\medskip }
\end{minipage} \\ \cdashline{2-2}

 & Status: \textbf{ Not Executed } \\ \hline

11 & Description \\
 & \begin{minipage}[t]{15cm}
{\footnotesize
In the EUI click the Start button.

\medskip }
\end{minipage}
\\ \cdashline{2-2}


 & Expected Result \\
 & \begin{minipage}[t]{15cm}{\footnotesize
N/A

\medskip }
\end{minipage} \\ \cdashline{2-2}

 & Actual Result \\
 & \begin{minipage}[t]{15cm}{\footnotesize

\medskip }
\end{minipage} \\ \cdashline{2-2}

 & Status: \textbf{ Not Executed } \\ \hline

12 & Description \\
 & \begin{minipage}[t]{15cm}
{\footnotesize
In the EUI click the Enable button.

\medskip }
\end{minipage}
\\ \cdashline{2-2}


 & Expected Result \\
 & \begin{minipage}[t]{15cm}{\footnotesize
N/A

\medskip }
\end{minipage} \\ \cdashline{2-2}

 & Actual Result \\
 & \begin{minipage}[t]{15cm}{\footnotesize

\medskip }
\end{minipage} \\ \cdashline{2-2}

 & Status: \textbf{ Not Executed } \\ \hline

13 & Description \\
 & \begin{minipage}[t]{15cm}
{\footnotesize
In the EUI click the Raise Mirror button.

\medskip }
\end{minipage}
\\ \cdashline{2-2}


 & Expected Result \\
 & \begin{minipage}[t]{15cm}{\footnotesize
N/A

\medskip }
\end{minipage} \\ \cdashline{2-2}

 & Actual Result \\
 & \begin{minipage}[t]{15cm}{\footnotesize

\medskip }
\end{minipage} \\ \cdashline{2-2}

 & Status: \textbf{ Not Executed } \\ \hline

14 & Description \\
 & \begin{minipage}[t]{15cm}
{\footnotesize
Wait until the detailed state reports ActiveState.

\medskip }
\end{minipage}
\\ \cdashline{2-2}


 & Expected Result \\
 & \begin{minipage}[t]{15cm}{\footnotesize
N/A

\medskip }
\end{minipage} \\ \cdashline{2-2}

 & Actual Result \\
 & \begin{minipage}[t]{15cm}{\footnotesize

\medskip }
\end{minipage} \\ \cdashline{2-2}

 & Status: \textbf{ Not Executed } \\ \hline

15 & Description \\
 & \begin{minipage}[t]{15cm}
{\footnotesize
Start collecting IMS data.

\medskip }
\end{minipage}
\\ \cdashline{2-2}


 & Expected Result \\
 & \begin{minipage}[t]{15cm}{\footnotesize
N/A

\medskip }
\end{minipage} \\ \cdashline{2-2}

 & Actual Result \\
 & \begin{minipage}[t]{15cm}{\footnotesize

\medskip }
\end{minipage} \\ \cdashline{2-2}

 & Status: \textbf{ Not Executed } \\ \hline

16 & Description \\
 & \begin{minipage}[t]{15cm}
{\footnotesize
In the controls cabinet flip the switch off for power subnet B.

\medskip }
\end{minipage}
\\ \cdashline{2-2}


 & Expected Result \\
 & \begin{minipage}[t]{15cm}{\footnotesize
N/A

\medskip }
\end{minipage} \\ \cdashline{2-2}

 & Actual Result \\
 & \begin{minipage}[t]{15cm}{\footnotesize

\medskip }
\end{minipage} \\ \cdashline{2-2}

 & Status: \textbf{ Not Executed } \\ \hline

17 & Description \\
 & \begin{minipage}[t]{15cm}
{\footnotesize
In the EUI verify an error occurs and an error code of 6110 / 1 is
observed. This error code indicates that ILC communication has been lost
to at least 1 ILC for over 60ms.

\medskip }
\end{minipage}
\\ \cdashline{2-2}


 & Expected Result \\
 & \begin{minipage}[t]{15cm}{\footnotesize
Error Code 6110 with detailed code of 1 is observed.

\medskip }
\end{minipage} \\ \cdashline{2-2}

 & Actual Result \\
 & \begin{minipage}[t]{15cm}{\footnotesize

\medskip }
\end{minipage} \\ \cdashline{2-2}

 & Status: \textbf{ Not Executed } \\ \hline

18 & Description \\
 & \begin{minipage}[t]{15cm}
{\footnotesize
Stop collecting IMS data.

\medskip }
\end{minipage}
\\ \cdashline{2-2}


 & Expected Result \\
 & \begin{minipage}[t]{15cm}{\footnotesize
N/A

\medskip }
\end{minipage} \\ \cdashline{2-2}

 & Actual Result \\
 & \begin{minipage}[t]{15cm}{\footnotesize

\medskip }
\end{minipage} \\ \cdashline{2-2}

 & Status: \textbf{ Not Executed } \\ \hline

19 & Description \\
 & \begin{minipage}[t]{15cm}
{\footnotesize
In the controls cabinet flip the switch on for power subnet B.

\medskip }
\end{minipage}
\\ \cdashline{2-2}


 & Expected Result \\
 & \begin{minipage}[t]{15cm}{\footnotesize
N/A

\medskip }
\end{minipage} \\ \cdashline{2-2}

 & Actual Result \\
 & \begin{minipage}[t]{15cm}{\footnotesize

\medskip }
\end{minipage} \\ \cdashline{2-2}

 & Status: \textbf{ Not Executed } \\ \hline

20 & Description \\
 & \begin{minipage}[t]{15cm}
{\footnotesize
In the EUI click the Standby button.

\medskip }
\end{minipage}
\\ \cdashline{2-2}


 & Expected Result \\
 & \begin{minipage}[t]{15cm}{\footnotesize
N/A

\medskip }
\end{minipage} \\ \cdashline{2-2}

 & Actual Result \\
 & \begin{minipage}[t]{15cm}{\footnotesize

\medskip }
\end{minipage} \\ \cdashline{2-2}

 & Status: \textbf{ Not Executed } \\ \hline

21 & Description \\
 & \begin{minipage}[t]{15cm}
{\footnotesize
In the EUI click the Start button.

\medskip }
\end{minipage}
\\ \cdashline{2-2}


 & Expected Result \\
 & \begin{minipage}[t]{15cm}{\footnotesize
N/A

\medskip }
\end{minipage} \\ \cdashline{2-2}

 & Actual Result \\
 & \begin{minipage}[t]{15cm}{\footnotesize

\medskip }
\end{minipage} \\ \cdashline{2-2}

 & Status: \textbf{ Not Executed } \\ \hline

22 & Description \\
 & \begin{minipage}[t]{15cm}
{\footnotesize
In the EUI click the Enable button.

\medskip }
\end{minipage}
\\ \cdashline{2-2}


 & Expected Result \\
 & \begin{minipage}[t]{15cm}{\footnotesize
N/A

\medskip }
\end{minipage} \\ \cdashline{2-2}

 & Actual Result \\
 & \begin{minipage}[t]{15cm}{\footnotesize

\medskip }
\end{minipage} \\ \cdashline{2-2}

 & Status: \textbf{ Not Executed } \\ \hline

23 & Description \\
 & \begin{minipage}[t]{15cm}
{\footnotesize
In the EUI click the Raise Mirror button.

\medskip }
\end{minipage}
\\ \cdashline{2-2}


 & Expected Result \\
 & \begin{minipage}[t]{15cm}{\footnotesize
N/A

\medskip }
\end{minipage} \\ \cdashline{2-2}

 & Actual Result \\
 & \begin{minipage}[t]{15cm}{\footnotesize

\medskip }
\end{minipage} \\ \cdashline{2-2}

 & Status: \textbf{ Not Executed } \\ \hline

24 & Description \\
 & \begin{minipage}[t]{15cm}
{\footnotesize
Wait until the detailed state reports ActiveState.

\medskip }
\end{minipage}
\\ \cdashline{2-2}


 & Expected Result \\
 & \begin{minipage}[t]{15cm}{\footnotesize
N/A

\medskip }
\end{minipage} \\ \cdashline{2-2}

 & Actual Result \\
 & \begin{minipage}[t]{15cm}{\footnotesize

\medskip }
\end{minipage} \\ \cdashline{2-2}

 & Status: \textbf{ Not Executed } \\ \hline

25 & Description \\
 & \begin{minipage}[t]{15cm}
{\footnotesize
In the EUI click the Lower Mirror button.

\medskip }
\end{minipage}
\\ \cdashline{2-2}


 & Expected Result \\
 & \begin{minipage}[t]{15cm}{\footnotesize
N/A

\medskip }
\end{minipage} \\ \cdashline{2-2}

 & Actual Result \\
 & \begin{minipage}[t]{15cm}{\footnotesize

\medskip }
\end{minipage} \\ \cdashline{2-2}

 & Status: \textbf{ Not Executed } \\ \hline

26 & Description \\
 & \begin{minipage}[t]{15cm}
{\footnotesize
Watch the IMS screen and wait for the Z position to change by at least
2mm.

\medskip }
\end{minipage}
\\ \cdashline{2-2}


 & Expected Result \\
 & \begin{minipage}[t]{15cm}{\footnotesize
N/A

\medskip }
\end{minipage} \\ \cdashline{2-2}

 & Actual Result \\
 & \begin{minipage}[t]{15cm}{\footnotesize

\medskip }
\end{minipage} \\ \cdashline{2-2}

 & Status: \textbf{ Not Executed } \\ \hline

27 & Description \\
 & \begin{minipage}[t]{15cm}
{\footnotesize
Start collecting IMS data.

\medskip }
\end{minipage}
\\ \cdashline{2-2}


 & Expected Result \\
 & \begin{minipage}[t]{15cm}{\footnotesize
N/A

\medskip }
\end{minipage} \\ \cdashline{2-2}

 & Actual Result \\
 & \begin{minipage}[t]{15cm}{\footnotesize

\medskip }
\end{minipage} \\ \cdashline{2-2}

 & Status: \textbf{ Not Executed } \\ \hline

28 & Description \\
 & \begin{minipage}[t]{15cm}
{\footnotesize
In the controls cabinet flip the switch off for power subnet C.

\medskip }
\end{minipage}
\\ \cdashline{2-2}


 & Expected Result \\
 & \begin{minipage}[t]{15cm}{\footnotesize
N/A

\medskip }
\end{minipage} \\ \cdashline{2-2}

 & Actual Result \\
 & \begin{minipage}[t]{15cm}{\footnotesize

\medskip }
\end{minipage} \\ \cdashline{2-2}

 & Status: \textbf{ Not Executed } \\ \hline

29 & Description \\
 & \begin{minipage}[t]{15cm}
{\footnotesize
In the EUI verify an error occurs and an error code of 6110 / 1 is
observed. This error code indicates that ILC communication has been lost
to at least 1 ILC for over 60ms.

\medskip }
\end{minipage}
\\ \cdashline{2-2}


 & Expected Result \\
 & \begin{minipage}[t]{15cm}{\footnotesize
Error Code 6110 with detailed code of 1 is observed.

\medskip }
\end{minipage} \\ \cdashline{2-2}

 & Actual Result \\
 & \begin{minipage}[t]{15cm}{\footnotesize

\medskip }
\end{minipage} \\ \cdashline{2-2}

 & Status: \textbf{ Not Executed } \\ \hline

30 & Description \\
 & \begin{minipage}[t]{15cm}
{\footnotesize
Stop collecting IMS data.

\medskip }
\end{minipage}
\\ \cdashline{2-2}


 & Expected Result \\
 & \begin{minipage}[t]{15cm}{\footnotesize
N/A

\medskip }
\end{minipage} \\ \cdashline{2-2}

 & Actual Result \\
 & \begin{minipage}[t]{15cm}{\footnotesize

\medskip }
\end{minipage} \\ \cdashline{2-2}

 & Status: \textbf{ Not Executed } \\ \hline

31 & Description \\
 & \begin{minipage}[t]{15cm}
{\footnotesize
In the controls cabinet flip the switch on for power subnet C.

\medskip }
\end{minipage}
\\ \cdashline{2-2}


 & Expected Result \\
 & \begin{minipage}[t]{15cm}{\footnotesize
N/A

\medskip }
\end{minipage} \\ \cdashline{2-2}

 & Actual Result \\
 & \begin{minipage}[t]{15cm}{\footnotesize

\medskip }
\end{minipage} \\ \cdashline{2-2}

 & Status: \textbf{ Not Executed } \\ \hline

32 & Description \\
 & \begin{minipage}[t]{15cm}
{\footnotesize
In the EUI click the Standby button.

\medskip }
\end{minipage}
\\ \cdashline{2-2}


 & Expected Result \\
 & \begin{minipage}[t]{15cm}{\footnotesize
N/A

\medskip }
\end{minipage} \\ \cdashline{2-2}

 & Actual Result \\
 & \begin{minipage}[t]{15cm}{\footnotesize

\medskip }
\end{minipage} \\ \cdashline{2-2}

 & Status: \textbf{ Not Executed } \\ \hline

33 & Description \\
 & \begin{minipage}[t]{15cm}
{\footnotesize
In the EUI click the Start button.

\medskip }
\end{minipage}
\\ \cdashline{2-2}


 & Expected Result \\
 & \begin{minipage}[t]{15cm}{\footnotesize
N/A

\medskip }
\end{minipage} \\ \cdashline{2-2}

 & Actual Result \\
 & \begin{minipage}[t]{15cm}{\footnotesize

\medskip }
\end{minipage} \\ \cdashline{2-2}

 & Status: \textbf{ Not Executed } \\ \hline

34 & Description \\
 & \begin{minipage}[t]{15cm}
{\footnotesize
In the EUI click the Enable button.

\medskip }
\end{minipage}
\\ \cdashline{2-2}


 & Expected Result \\
 & \begin{minipage}[t]{15cm}{\footnotesize
N/A

\medskip }
\end{minipage} \\ \cdashline{2-2}

 & Actual Result \\
 & \begin{minipage}[t]{15cm}{\footnotesize

\medskip }
\end{minipage} \\ \cdashline{2-2}

 & Status: \textbf{ Not Executed } \\ \hline

35 & Description \\
 & \begin{minipage}[t]{15cm}
{\footnotesize
Start collecting IMS data.

\medskip }
\end{minipage}
\\ \cdashline{2-2}


 & Expected Result \\
 & \begin{minipage}[t]{15cm}{\footnotesize
N/A

\medskip }
\end{minipage} \\ \cdashline{2-2}

 & Actual Result \\
 & \begin{minipage}[t]{15cm}{\footnotesize

\medskip }
\end{minipage} \\ \cdashline{2-2}

 & Status: \textbf{ Not Executed } \\ \hline

36 & Description \\
 & \begin{minipage}[t]{15cm}
{\footnotesize
In the controls cabinet flip the switch off for power subnet D.

\medskip }
\end{minipage}
\\ \cdashline{2-2}


 & Expected Result \\
 & \begin{minipage}[t]{15cm}{\footnotesize
N/A

\medskip }
\end{minipage} \\ \cdashline{2-2}

 & Actual Result \\
 & \begin{minipage}[t]{15cm}{\footnotesize

\medskip }
\end{minipage} \\ \cdashline{2-2}

 & Status: \textbf{ Not Executed } \\ \hline

37 & Description \\
 & \begin{minipage}[t]{15cm}
{\footnotesize
In the EUI verify an error occurs and an error code of 6110 / 1 is
observed. This error code indicates that ILC communication has been lost
to at least 1 ILC for over 60ms.

\medskip }
\end{minipage}
\\ \cdashline{2-2}


 & Expected Result \\
 & \begin{minipage}[t]{15cm}{\footnotesize
Error Code 6110 with detailed code of 1 is observed.

\medskip }
\end{minipage} \\ \cdashline{2-2}

 & Actual Result \\
 & \begin{minipage}[t]{15cm}{\footnotesize

\medskip }
\end{minipage} \\ \cdashline{2-2}

 & Status: \textbf{ Not Executed } \\ \hline

38 & Description \\
 & \begin{minipage}[t]{15cm}
{\footnotesize
Stop collecting IMS data.

\medskip }
\end{minipage}
\\ \cdashline{2-2}


 & Expected Result \\
 & \begin{minipage}[t]{15cm}{\footnotesize
N/A

\medskip }
\end{minipage} \\ \cdashline{2-2}

 & Actual Result \\
 & \begin{minipage}[t]{15cm}{\footnotesize

\medskip }
\end{minipage} \\ \cdashline{2-2}

 & Status: \textbf{ Not Executed } \\ \hline

39 & Description \\
 & \begin{minipage}[t]{15cm}
{\footnotesize
In the controls cabinet flip the switch on for power subnet D.

\medskip }
\end{minipage}
\\ \cdashline{2-2}


 & Expected Result \\
 & \begin{minipage}[t]{15cm}{\footnotesize
N/A

\medskip }
\end{minipage} \\ \cdashline{2-2}

 & Actual Result \\
 & \begin{minipage}[t]{15cm}{\footnotesize

\medskip }
\end{minipage} \\ \cdashline{2-2}

 & Status: \textbf{ Not Executed } \\ \hline

40 & Description \\
 & \begin{minipage}[t]{15cm}
{\footnotesize
In the EUI click the Standby button.

\medskip }
\end{minipage}
\\ \cdashline{2-2}


 & Expected Result \\
 & \begin{minipage}[t]{15cm}{\footnotesize
N/A

\medskip }
\end{minipage} \\ \cdashline{2-2}

 & Actual Result \\
 & \begin{minipage}[t]{15cm}{\footnotesize

\medskip }
\end{minipage} \\ \cdashline{2-2}

 & Status: \textbf{ Not Executed } \\ \hline

41 & Description \\
 & \begin{minipage}[t]{15cm}
{\footnotesize
In the EUI click the Start button.

\medskip }
\end{minipage}
\\ \cdashline{2-2}


 & Expected Result \\
 & \begin{minipage}[t]{15cm}{\footnotesize
N/A

\medskip }
\end{minipage} \\ \cdashline{2-2}

 & Actual Result \\
 & \begin{minipage}[t]{15cm}{\footnotesize

\medskip }
\end{minipage} \\ \cdashline{2-2}

 & Status: \textbf{ Not Executed } \\ \hline

42 & Description \\
 & \begin{minipage}[t]{15cm}
{\footnotesize
In the EUI click the Enable button.

\medskip }
\end{minipage}
\\ \cdashline{2-2}


 & Expected Result \\
 & \begin{minipage}[t]{15cm}{\footnotesize
N/A

\medskip }
\end{minipage} \\ \cdashline{2-2}

 & Actual Result \\
 & \begin{minipage}[t]{15cm}{\footnotesize

\medskip }
\end{minipage} \\ \cdashline{2-2}

 & Status: \textbf{ Not Executed } \\ \hline

43 & Description \\
 & \begin{minipage}[t]{15cm}
{\footnotesize
In the EUI click the Enter Engineering button.

\medskip }
\end{minipage}
\\ \cdashline{2-2}


 & Expected Result \\
 & \begin{minipage}[t]{15cm}{\footnotesize
N/A

\medskip }
\end{minipage} \\ \cdashline{2-2}

 & Actual Result \\
 & \begin{minipage}[t]{15cm}{\footnotesize

\medskip }
\end{minipage} \\ \cdashline{2-2}

 & Status: \textbf{ Not Executed } \\ \hline

44 & Description \\
 & \begin{minipage}[t]{15cm}
{\footnotesize
In the EUI click the Raise Mirror button.

\medskip }
\end{minipage}
\\ \cdashline{2-2}


 & Expected Result \\
 & \begin{minipage}[t]{15cm}{\footnotesize
N/A

\medskip }
\end{minipage} \\ \cdashline{2-2}

 & Actual Result \\
 & \begin{minipage}[t]{15cm}{\footnotesize

\medskip }
\end{minipage} \\ \cdashline{2-2}

 & Status: \textbf{ Not Executed } \\ \hline

45 & Description \\
 & \begin{minipage}[t]{15cm}
{\footnotesize
Watch the IMS screen and wait for the Z position to change by at least
2mm.

\medskip }
\end{minipage}
\\ \cdashline{2-2}


 & Expected Result \\
 & \begin{minipage}[t]{15cm}{\footnotesize
N/A

\medskip }
\end{minipage} \\ \cdashline{2-2}

 & Actual Result \\
 & \begin{minipage}[t]{15cm}{\footnotesize

\medskip }
\end{minipage} \\ \cdashline{2-2}

 & Status: \textbf{ Not Executed } \\ \hline

46 & Description \\
 & \begin{minipage}[t]{15cm}
{\footnotesize
Start collecting IMS data.

\medskip }
\end{minipage}
\\ \cdashline{2-2}


 & Expected Result \\
 & \begin{minipage}[t]{15cm}{\footnotesize
N/A

\medskip }
\end{minipage} \\ \cdashline{2-2}

 & Actual Result \\
 & \begin{minipage}[t]{15cm}{\footnotesize

\medskip }
\end{minipage} \\ \cdashline{2-2}

 & Status: \textbf{ Not Executed } \\ \hline

47 & Description \\
 & \begin{minipage}[t]{15cm}
{\footnotesize
In the controls cabinet flip the switch off for power subnet A.

\medskip }
\end{minipage}
\\ \cdashline{2-2}


 & Expected Result \\
 & \begin{minipage}[t]{15cm}{\footnotesize
N/A

\medskip }
\end{minipage} \\ \cdashline{2-2}

 & Actual Result \\
 & \begin{minipage}[t]{15cm}{\footnotesize

\medskip }
\end{minipage} \\ \cdashline{2-2}

 & Status: \textbf{ Not Executed } \\ \hline

48 & Description \\
 & \begin{minipage}[t]{15cm}
{\footnotesize
In the EUI verify an error occurs and an error code of 6110 / 1 is
observed. This error code indicates that ILC communication has been lost
to at least 1 ILC for over 60ms.

\medskip }
\end{minipage}
\\ \cdashline{2-2}


 & Expected Result \\
 & \begin{minipage}[t]{15cm}{\footnotesize
Error Code 6110 with detailed code of 1 is observed.

\medskip }
\end{minipage} \\ \cdashline{2-2}

 & Actual Result \\
 & \begin{minipage}[t]{15cm}{\footnotesize

\medskip }
\end{minipage} \\ \cdashline{2-2}

 & Status: \textbf{ Not Executed } \\ \hline

49 & Description \\
 & \begin{minipage}[t]{15cm}
{\footnotesize
Stop collecting IMS data.

\medskip }
\end{minipage}
\\ \cdashline{2-2}


 & Expected Result \\
 & \begin{minipage}[t]{15cm}{\footnotesize
N/A

\medskip }
\end{minipage} \\ \cdashline{2-2}

 & Actual Result \\
 & \begin{minipage}[t]{15cm}{\footnotesize

\medskip }
\end{minipage} \\ \cdashline{2-2}

 & Status: \textbf{ Not Executed } \\ \hline

50 & Description \\
 & \begin{minipage}[t]{15cm}
{\footnotesize
In the controls cabinet flip the switch on for power subnet A.

\medskip }
\end{minipage}
\\ \cdashline{2-2}


 & Expected Result \\
 & \begin{minipage}[t]{15cm}{\footnotesize
N/A

\medskip }
\end{minipage} \\ \cdashline{2-2}

 & Actual Result \\
 & \begin{minipage}[t]{15cm}{\footnotesize

\medskip }
\end{minipage} \\ \cdashline{2-2}

 & Status: \textbf{ Not Executed } \\ \hline

51 & Description \\
 & \begin{minipage}[t]{15cm}
{\footnotesize
In the EUI click the Standby button.

\medskip }
\end{minipage}
\\ \cdashline{2-2}


 & Expected Result \\
 & \begin{minipage}[t]{15cm}{\footnotesize
N/A

\medskip }
\end{minipage} \\ \cdashline{2-2}

 & Actual Result \\
 & \begin{minipage}[t]{15cm}{\footnotesize

\medskip }
\end{minipage} \\ \cdashline{2-2}

 & Status: \textbf{ Not Executed } \\ \hline

52 & Description \\
 & \begin{minipage}[t]{15cm}
{\footnotesize
In the EUI click the Start button.

\medskip }
\end{minipage}
\\ \cdashline{2-2}


 & Expected Result \\
 & \begin{minipage}[t]{15cm}{\footnotesize
N/A

\medskip }
\end{minipage} \\ \cdashline{2-2}

 & Actual Result \\
 & \begin{minipage}[t]{15cm}{\footnotesize

\medskip }
\end{minipage} \\ \cdashline{2-2}

 & Status: \textbf{ Not Executed } \\ \hline

53 & Description \\
 & \begin{minipage}[t]{15cm}
{\footnotesize
In the EUI click the Enable button.

\medskip }
\end{minipage}
\\ \cdashline{2-2}


 & Expected Result \\
 & \begin{minipage}[t]{15cm}{\footnotesize
N/A

\medskip }
\end{minipage} \\ \cdashline{2-2}

 & Actual Result \\
 & \begin{minipage}[t]{15cm}{\footnotesize

\medskip }
\end{minipage} \\ \cdashline{2-2}

 & Status: \textbf{ Not Executed } \\ \hline

54 & Description \\
 & \begin{minipage}[t]{15cm}
{\footnotesize
In the EUI click the Enter Engineering button.

\medskip }
\end{minipage}
\\ \cdashline{2-2}


 & Expected Result \\
 & \begin{minipage}[t]{15cm}{\footnotesize
N/A

\medskip }
\end{minipage} \\ \cdashline{2-2}

 & Actual Result \\
 & \begin{minipage}[t]{15cm}{\footnotesize

\medskip }
\end{minipage} \\ \cdashline{2-2}

 & Status: \textbf{ Not Executed } \\ \hline

55 & Description \\
 & \begin{minipage}[t]{15cm}
{\footnotesize
In the EUI click the Raise Mirror button.

\medskip }
\end{minipage}
\\ \cdashline{2-2}


 & Expected Result \\
 & \begin{minipage}[t]{15cm}{\footnotesize
N/A

\medskip }
\end{minipage} \\ \cdashline{2-2}

 & Actual Result \\
 & \begin{minipage}[t]{15cm}{\footnotesize

\medskip }
\end{minipage} \\ \cdashline{2-2}

 & Status: \textbf{ Not Executed } \\ \hline

56 & Description \\
 & \begin{minipage}[t]{15cm}
{\footnotesize
Wait until the detailed state reports ActiveEngineeringState.

\medskip }
\end{minipage}
\\ \cdashline{2-2}


 & Expected Result \\
 & \begin{minipage}[t]{15cm}{\footnotesize
N/A

\medskip }
\end{minipage} \\ \cdashline{2-2}

 & Actual Result \\
 & \begin{minipage}[t]{15cm}{\footnotesize

\medskip }
\end{minipage} \\ \cdashline{2-2}

 & Status: \textbf{ Not Executed } \\ \hline

57 & Description \\
 & \begin{minipage}[t]{15cm}
{\footnotesize
Start collecting IMS data.

\medskip }
\end{minipage}
\\ \cdashline{2-2}


 & Expected Result \\
 & \begin{minipage}[t]{15cm}{\footnotesize
N/A

\medskip }
\end{minipage} \\ \cdashline{2-2}

 & Actual Result \\
 & \begin{minipage}[t]{15cm}{\footnotesize

\medskip }
\end{minipage} \\ \cdashline{2-2}

 & Status: \textbf{ Not Executed } \\ \hline

58 & Description \\
 & \begin{minipage}[t]{15cm}
{\footnotesize
In the controls cabinet flip the switch off for power subnet B.

\medskip }
\end{minipage}
\\ \cdashline{2-2}


 & Expected Result \\
 & \begin{minipage}[t]{15cm}{\footnotesize
N/A

\medskip }
\end{minipage} \\ \cdashline{2-2}

 & Actual Result \\
 & \begin{minipage}[t]{15cm}{\footnotesize

\medskip }
\end{minipage} \\ \cdashline{2-2}

 & Status: \textbf{ Not Executed } \\ \hline

59 & Description \\
 & \begin{minipage}[t]{15cm}
{\footnotesize
In the EUI verify an error occurs and an error code of 6110 / 1 is
observed. This error code indicates that ILC communication has been lost
to at least 1 ILC for over 60ms.

\medskip }
\end{minipage}
\\ \cdashline{2-2}


 & Expected Result \\
 & \begin{minipage}[t]{15cm}{\footnotesize
Error Code 6110 with detailed code of 1 is observed.

\medskip }
\end{minipage} \\ \cdashline{2-2}

 & Actual Result \\
 & \begin{minipage}[t]{15cm}{\footnotesize

\medskip }
\end{minipage} \\ \cdashline{2-2}

 & Status: \textbf{ Not Executed } \\ \hline

60 & Description \\
 & \begin{minipage}[t]{15cm}
{\footnotesize
Stop collecting IMS data.

\medskip }
\end{minipage}
\\ \cdashline{2-2}


 & Expected Result \\
 & \begin{minipage}[t]{15cm}{\footnotesize
N/A

\medskip }
\end{minipage} \\ \cdashline{2-2}

 & Actual Result \\
 & \begin{minipage}[t]{15cm}{\footnotesize

\medskip }
\end{minipage} \\ \cdashline{2-2}

 & Status: \textbf{ Not Executed } \\ \hline

61 & Description \\
 & \begin{minipage}[t]{15cm}
{\footnotesize
In the controls cabinet flip the switch on for power subnet B.

\medskip }
\end{minipage}
\\ \cdashline{2-2}


 & Expected Result \\
 & \begin{minipage}[t]{15cm}{\footnotesize
N/A

\medskip }
\end{minipage} \\ \cdashline{2-2}

 & Actual Result \\
 & \begin{minipage}[t]{15cm}{\footnotesize

\medskip }
\end{minipage} \\ \cdashline{2-2}

 & Status: \textbf{ Not Executed } \\ \hline

62 & Description \\
 & \begin{minipage}[t]{15cm}
{\footnotesize
In the EUI click the Standby button.

\medskip }
\end{minipage}
\\ \cdashline{2-2}


 & Expected Result \\
 & \begin{minipage}[t]{15cm}{\footnotesize
N/A

\medskip }
\end{minipage} \\ \cdashline{2-2}

 & Actual Result \\
 & \begin{minipage}[t]{15cm}{\footnotesize

\medskip }
\end{minipage} \\ \cdashline{2-2}

 & Status: \textbf{ Not Executed } \\ \hline

63 & Description \\
 & \begin{minipage}[t]{15cm}
{\footnotesize
In the EUI click the Start button.

\medskip }
\end{minipage}
\\ \cdashline{2-2}


 & Expected Result \\
 & \begin{minipage}[t]{15cm}{\footnotesize
N/A

\medskip }
\end{minipage} \\ \cdashline{2-2}

 & Actual Result \\
 & \begin{minipage}[t]{15cm}{\footnotesize

\medskip }
\end{minipage} \\ \cdashline{2-2}

 & Status: \textbf{ Not Executed } \\ \hline

64 & Description \\
 & \begin{minipage}[t]{15cm}
{\footnotesize
In the EUI click the Enable button.

\medskip }
\end{minipage}
\\ \cdashline{2-2}


 & Expected Result \\
 & \begin{minipage}[t]{15cm}{\footnotesize
N/A

\medskip }
\end{minipage} \\ \cdashline{2-2}

 & Actual Result \\
 & \begin{minipage}[t]{15cm}{\footnotesize

\medskip }
\end{minipage} \\ \cdashline{2-2}

 & Status: \textbf{ Not Executed } \\ \hline

65 & Description \\
 & \begin{minipage}[t]{15cm}
{\footnotesize
In the EUI click the Enter Engineering button.

\medskip }
\end{minipage}
\\ \cdashline{2-2}


 & Expected Result \\
 & \begin{minipage}[t]{15cm}{\footnotesize
N/A

\medskip }
\end{minipage} \\ \cdashline{2-2}

 & Actual Result \\
 & \begin{minipage}[t]{15cm}{\footnotesize

\medskip }
\end{minipage} \\ \cdashline{2-2}

 & Status: \textbf{ Not Executed } \\ \hline

66 & Description \\
 & \begin{minipage}[t]{15cm}
{\footnotesize
In the EUI click the Raise Mirror button.

\medskip }
\end{minipage}
\\ \cdashline{2-2}


 & Expected Result \\
 & \begin{minipage}[t]{15cm}{\footnotesize
N/A

\medskip }
\end{minipage} \\ \cdashline{2-2}

 & Actual Result \\
 & \begin{minipage}[t]{15cm}{\footnotesize

\medskip }
\end{minipage} \\ \cdashline{2-2}

 & Status: \textbf{ Not Executed } \\ \hline

67 & Description \\
 & \begin{minipage}[t]{15cm}
{\footnotesize
Wait until the detailed state reports ActiveEngineeringState.

\medskip }
\end{minipage}
\\ \cdashline{2-2}


 & Expected Result \\
 & \begin{minipage}[t]{15cm}{\footnotesize
N/A

\medskip }
\end{minipage} \\ \cdashline{2-2}

 & Actual Result \\
 & \begin{minipage}[t]{15cm}{\footnotesize

\medskip }
\end{minipage} \\ \cdashline{2-2}

 & Status: \textbf{ Not Executed } \\ \hline

68 & Description \\
 & \begin{minipage}[t]{15cm}
{\footnotesize
In the EUI click the Lower Mirror button.

\medskip }
\end{minipage}
\\ \cdashline{2-2}


 & Expected Result \\
 & \begin{minipage}[t]{15cm}{\footnotesize
N/A

\medskip }
\end{minipage} \\ \cdashline{2-2}

 & Actual Result \\
 & \begin{minipage}[t]{15cm}{\footnotesize

\medskip }
\end{minipage} \\ \cdashline{2-2}

 & Status: \textbf{ Not Executed } \\ \hline

69 & Description \\
 & \begin{minipage}[t]{15cm}
{\footnotesize
Watch the IMS screen and wait for the Z position to change by at least
2mm.

\medskip }
\end{minipage}
\\ \cdashline{2-2}


 & Expected Result \\
 & \begin{minipage}[t]{15cm}{\footnotesize
N/A

\medskip }
\end{minipage} \\ \cdashline{2-2}

 & Actual Result \\
 & \begin{minipage}[t]{15cm}{\footnotesize

\medskip }
\end{minipage} \\ \cdashline{2-2}

 & Status: \textbf{ Not Executed } \\ \hline

70 & Description \\
 & \begin{minipage}[t]{15cm}
{\footnotesize
Start collecting IMS data.

\medskip }
\end{minipage}
\\ \cdashline{2-2}


 & Expected Result \\
 & \begin{minipage}[t]{15cm}{\footnotesize
N/A

\medskip }
\end{minipage} \\ \cdashline{2-2}

 & Actual Result \\
 & \begin{minipage}[t]{15cm}{\footnotesize

\medskip }
\end{minipage} \\ \cdashline{2-2}

 & Status: \textbf{ Not Executed } \\ \hline

71 & Description \\
 & \begin{minipage}[t]{15cm}
{\footnotesize
In the controls cabinet flip the switch off for power subnet C.

\medskip }
\end{minipage}
\\ \cdashline{2-2}


 & Expected Result \\
 & \begin{minipage}[t]{15cm}{\footnotesize
N/A

\medskip }
\end{minipage} \\ \cdashline{2-2}

 & Actual Result \\
 & \begin{minipage}[t]{15cm}{\footnotesize

\medskip }
\end{minipage} \\ \cdashline{2-2}

 & Status: \textbf{ Not Executed } \\ \hline

72 & Description \\
 & \begin{minipage}[t]{15cm}
{\footnotesize
In the EUI verify an error occurs and an error code of 6110 / 1 is
observed. This error code indicates that ILC communication has been lost
to at least 1 ILC for over 60ms.

\medskip }
\end{minipage}
\\ \cdashline{2-2}


 & Expected Result \\
 & \begin{minipage}[t]{15cm}{\footnotesize
Error Code 6110 with detailed code of 1 is observed.

\medskip }
\end{minipage} \\ \cdashline{2-2}

 & Actual Result \\
 & \begin{minipage}[t]{15cm}{\footnotesize

\medskip }
\end{minipage} \\ \cdashline{2-2}

 & Status: \textbf{ Not Executed } \\ \hline

73 & Description \\
 & \begin{minipage}[t]{15cm}
{\footnotesize
Stop collecting IMS data.

\medskip }
\end{minipage}
\\ \cdashline{2-2}


 & Expected Result \\
 & \begin{minipage}[t]{15cm}{\footnotesize
N/A

\medskip }
\end{minipage} \\ \cdashline{2-2}

 & Actual Result \\
 & \begin{minipage}[t]{15cm}{\footnotesize

\medskip }
\end{minipage} \\ \cdashline{2-2}

 & Status: \textbf{ Not Executed } \\ \hline

74 & Description \\
 & \begin{minipage}[t]{15cm}
{\footnotesize
In the controls cabinet flip the switch on for power subnet C.

\medskip }
\end{minipage}
\\ \cdashline{2-2}


 & Expected Result \\
 & \begin{minipage}[t]{15cm}{\footnotesize
N/A

\medskip }
\end{minipage} \\ \cdashline{2-2}

 & Actual Result \\
 & \begin{minipage}[t]{15cm}{\footnotesize

\medskip }
\end{minipage} \\ \cdashline{2-2}

 & Status: \textbf{ Not Executed } \\ \hline

75 & Description \\
 & \begin{minipage}[t]{15cm}
{\footnotesize
In the EUI click the Standby button.

\medskip }
\end{minipage}
\\ \cdashline{2-2}


 & Expected Result \\
 & \begin{minipage}[t]{15cm}{\footnotesize
N/A

\medskip }
\end{minipage} \\ \cdashline{2-2}

 & Actual Result \\
 & \begin{minipage}[t]{15cm}{\footnotesize

\medskip }
\end{minipage} \\ \cdashline{2-2}

 & Status: \textbf{ Not Executed } \\ \hline

76 & Description \\
 & \begin{minipage}[t]{15cm}
{\footnotesize
In the EUI click the Start button.

\medskip }
\end{minipage}
\\ \cdashline{2-2}


 & Expected Result \\
 & \begin{minipage}[t]{15cm}{\footnotesize
N/A

\medskip }
\end{minipage} \\ \cdashline{2-2}

 & Actual Result \\
 & \begin{minipage}[t]{15cm}{\footnotesize

\medskip }
\end{minipage} \\ \cdashline{2-2}

 & Status: \textbf{ Not Executed } \\ \hline

77 & Description \\
 & \begin{minipage}[t]{15cm}
{\footnotesize
In the EUI click the Enable button.

\medskip }
\end{minipage}
\\ \cdashline{2-2}


 & Expected Result \\
 & \begin{minipage}[t]{15cm}{\footnotesize
N/A

\medskip }
\end{minipage} \\ \cdashline{2-2}

 & Actual Result \\
 & \begin{minipage}[t]{15cm}{\footnotesize

\medskip }
\end{minipage} \\ \cdashline{2-2}

 & Status: \textbf{ Not Executed } \\ \hline

78 & Description \\
 & \begin{minipage}[t]{15cm}
{\footnotesize
In the EUI click the Enter Engineering button.

\medskip }
\end{minipage}
\\ \cdashline{2-2}


 & Expected Result \\
 & \begin{minipage}[t]{15cm}{\footnotesize
N/A

\medskip }
\end{minipage} \\ \cdashline{2-2}

 & Actual Result \\
 & \begin{minipage}[t]{15cm}{\footnotesize

\medskip }
\end{minipage} \\ \cdashline{2-2}

 & Status: \textbf{ Not Executed } \\ \hline

79 & Description \\
 & \begin{minipage}[t]{15cm}
{\footnotesize
Start collecting IMS data.

\medskip }
\end{minipage}
\\ \cdashline{2-2}


 & Expected Result \\
 & \begin{minipage}[t]{15cm}{\footnotesize
N/A

\medskip }
\end{minipage} \\ \cdashline{2-2}

 & Actual Result \\
 & \begin{minipage}[t]{15cm}{\footnotesize

\medskip }
\end{minipage} \\ \cdashline{2-2}

 & Status: \textbf{ Not Executed } \\ \hline

80 & Description \\
 & \begin{minipage}[t]{15cm}
{\footnotesize
In the controls cabinet flip the switch off for power subnet D.

\medskip }
\end{minipage}
\\ \cdashline{2-2}


 & Expected Result \\
 & \begin{minipage}[t]{15cm}{\footnotesize
N/A

\medskip }
\end{minipage} \\ \cdashline{2-2}

 & Actual Result \\
 & \begin{minipage}[t]{15cm}{\footnotesize

\medskip }
\end{minipage} \\ \cdashline{2-2}

 & Status: \textbf{ Not Executed } \\ \hline

81 & Description \\
 & \begin{minipage}[t]{15cm}
{\footnotesize
In the EUI verify an error occurs and an error code of 6110 / 1 is
observed. This error code indicates that ILC communication has been lost
to at least 1 ILC for over 60ms.

\medskip }
\end{minipage}
\\ \cdashline{2-2}


 & Expected Result \\
 & \begin{minipage}[t]{15cm}{\footnotesize
Error Code 6110 with detailed code of 1 is observed.

\medskip }
\end{minipage} \\ \cdashline{2-2}

 & Actual Result \\
 & \begin{minipage}[t]{15cm}{\footnotesize

\medskip }
\end{minipage} \\ \cdashline{2-2}

 & Status: \textbf{ Not Executed } \\ \hline

82 & Description \\
 & \begin{minipage}[t]{15cm}
{\footnotesize
Stop collecting IMS data.

\medskip }
\end{minipage}
\\ \cdashline{2-2}


 & Expected Result \\
 & \begin{minipage}[t]{15cm}{\footnotesize
N/A

\medskip }
\end{minipage} \\ \cdashline{2-2}

 & Actual Result \\
 & \begin{minipage}[t]{15cm}{\footnotesize

\medskip }
\end{minipage} \\ \cdashline{2-2}

 & Status: \textbf{ Not Executed } \\ \hline

83 & Description \\
 & \begin{minipage}[t]{15cm}
{\footnotesize
In the controls cabinet flip the switch on for power subnet D.

\medskip }
\end{minipage}
\\ \cdashline{2-2}


 & Expected Result \\
 & \begin{minipage}[t]{15cm}{\footnotesize
N/A

\medskip }
\end{minipage} \\ \cdashline{2-2}

 & Actual Result \\
 & \begin{minipage}[t]{15cm}{\footnotesize

\medskip }
\end{minipage} \\ \cdashline{2-2}

 & Status: \textbf{ Not Executed } \\ \hline

84 & Description \\
 & \begin{minipage}[t]{15cm}
{\footnotesize
In the EUI click the Standby button.

\medskip }
\end{minipage}
\\ \cdashline{2-2}


 & Expected Result \\
 & \begin{minipage}[t]{15cm}{\footnotesize
N/A

\medskip }
\end{minipage} \\ \cdashline{2-2}

 & Actual Result \\
 & \begin{minipage}[t]{15cm}{\footnotesize

\medskip }
\end{minipage} \\ \cdashline{2-2}

 & Status: \textbf{ Not Executed } \\ \hline

85 & Description \\
 & \begin{minipage}[t]{15cm}
{\footnotesize
Compress all of the test data into a zip file called
M13T031-yyyymmddhhnn.zip and attach it to this step.

\medskip }
\end{minipage}
\\ \cdashline{2-2}


 & Expected Result \\
 & \begin{minipage}[t]{15cm}{\footnotesize
N/A

\medskip }
\end{minipage} \\ \cdashline{2-2}

 & Actual Result \\
 & \begin{minipage}[t]{15cm}{\footnotesize

\medskip }
\end{minipage} \\ \cdashline{2-2}

 & Status: \textbf{ Not Executed } \\ \hline

86 & Description \\
 & \begin{minipage}[t]{15cm}
{\footnotesize
Verify the fall rate for all test data doesn't exceed 15mm/s.

\medskip }
\end{minipage}
\\ \cdashline{2-2}


 & Expected Result \\
 & \begin{minipage}[t]{15cm}{\footnotesize
All fall rates don't exceed 15mm/s

\medskip }
\end{minipage} \\ \cdashline{2-2}

 & Actual Result \\
 & \begin{minipage}[t]{15cm}{\footnotesize

\medskip }
\end{minipage} \\ \cdashline{2-2}

 & Status: \textbf{ Not Executed } \\ \hline

\end{longtable}


\newpage
\appendix
%Make sure lsst-texmf/bin/generateAcronyms.py is in your path
\section{Acronyms used in this document}\label{sec:acronyms}
\input{acronyms.tex}

\end{document}
